%%%%%%%%%%%%%%%%%%%%%%%%%%%%%%%%%%%%%%%%%%%%%%%%%%%%%%%%%%%%%%%%%%%%%%%%%%%%
%% Author template for Management Science (mnsc) for articles with e-companion (EC)
%% Mirko Janc, Ph.D., INFORMS, mirko.janc@informs.org
%% ver. 0.95, December 2010
%%%%%%%%%%%%%%%%%%%%%%%%%%%%%%%%%%%%%%%%%%%%%%%%%%%%%%%%%%%%%%%%%%%%%%%%%%%%
\documentclass[mnsc,blindrev]{informs3} % current default for manuscript submission
%\documentclass[mnsc,nonblindrev]{informs3}

\OneAndAHalfSpacedXI % current default line spacing
%%\OneAndAHalfSpacedXII 
%%\DoubleSpacedXII
%%\DoubleSpacedXI

% If hyperref is used, dvi-to-ps driver of choice must be declared as
%   an additional option to the \documentstyle. For example
%\documentclass[dvips,mnsc]{informs3}      % if dvips is used
%\documentclass[dvipsone,mnsc]{informs3}   % if dvipsone is used, etc.

% Private macros here (check that there is no clash with the style)
\usepackage{graphicx}

% Natbib setup for author-year style
\usepackage{natbib}
 \bibpunct[, ]{(}{)}{,}{a}{}{,}%
 \def\bibfont{\small}%
 \def\bibsep{\smallskipamount}%
 \def\bibhang{24pt}%
 \def\newblock{\ }%
 \def\BIBand{and}%

%% Setup of theorem styles. Outcomment only one.
%% Preferred default is the first option.
\TheoremsNumberedThrough     % Preferred (Theorem 1, Lemma 1, Theorem 2)
%\TheoremsNumberedByChapter  % (Theorem 1.1, Lema 1.1, Theorem 1.2)
\ECRepeatTheorems

%% Setup of the equation numbering system. Outcomment only one.
%% Preferred default is the first option.
\EquationsNumberedThrough    % Default: (1), (2), ...
%\EquationsNumberedBySection % (1.1), (1.2), ...

% For new submissions, leave this number blank.
% For revisions, input the manuscript number assigned by the on-line
% system along with a suffix ".Rx" where x is the revision number.
\MANUSCRIPTNO{}

%%%%%%%%%%%%%%%%
\begin{document}
%%%%%%%%%%%%%%%%

% Outcomment only when entries are known. Otherwise leave as is and
%   default values will be used.
%\setcounter{page}{1}
%\VOLUME{00}%
%\NO{0}%
%\MONTH{Xxxxx}% (month or a similar seasonal id)
%\YEAR{0000}% e.g., 2005
%\FIRSTPAGE{000}%
%\LASTPAGE{000}%
%\SHORTYEAR{00}% shortened year (two-digit)
%\ISSUE{0000} %
%\LONGFIRSTPAGE{0001} %
%\DOI{10.1287/xxxx.0000.0000}%

% Author's names for the running heads
% Sample depending on the number of authors;
% \RUNAUTHOR{Jones}
% \RUNAUTHOR{Jones and Wilson}
% \RUNAUTHOR{Jones, Miller, and Wilson}
% \RUNAUTHOR{Jones et al.} % for four or more authors
% Enter authors following the given pattern:
%\RUNAUTHOR{}

% Title or shortened title suitable for running heads. Sample:
% \RUNTITLE{Bundling Information Goods of Decreasing Value}
% Enter the (shortened) title:
\RUNTITLE{Signal and Noise of Reviews}

% Full title. Sample:
% \TITLE{Bundling Information Goods of Decreasing Value}
% Enter the full title:
\TITLE{The Signal and Noise of Performance Reviews in an Online Market Place: The Case of Residential Solar Installations}

% Block of authors and their affiliations starts here:
% NOTE: Authors with same affiliation, if the order of authors allows,
%   should be entered in ONE field, separated by a comma.
%   \EMAIL field can be repeated if more than one author
\ARTICLEAUTHORS{%
\AUTHOR{Snidely Slippery}
\AFF{Department of Bread Spread Engineering, Dairy University, Cowtown, IL 60208, \EMAIL{slippery@dairy.edu}} %, \URL{}}
\AUTHOR{Marg Arinella}
\AFF{Institute for Food Adulteration, University of Food Plains, Food Plains, MN 55599, \EMAIL{m.arinella@adult.ufp.edu}}
% Enter all authors
} % end of the block

\ABSTRACT{%
This paper  
% Enter your abstract
}%

% Sample
%\KEYWORDS{deterministic inventory theory; infinite linear programming duality;
%  existence of optimal policies; semi-Markov decision process; cyclic schedule}

% Fill in data. If unknown, outcomment the field
\KEYWORDS{marketplace, reviews} \HISTORY{Update: November, 2019}

\maketitle
%%%%%%%%%%%%%%%%%%%%%%%%%%%%%%%%%%%%%%%%%%%%%%%%%%%%%%%%%%%%%%%%%%%%%%

% Samples of sectioning (and labeling) in MNSC
% NOTE: (1) \section and \subsection do NOT end with a period
%       (2) \subsubsection and lower need end punctuation
%       (3) capitalization is as shown (title style).
%
%\section{Introduction.}\label{intro} %%1.
%\subsection{Duality and the Classical EOQ Problem.}\label{class-EOQ} %% 1.1.
%\subsection{Outline.}\label{outline1} %% 1.2.
%\subsubsection{Cyclic Schedules for the General Deterministic SMDP.}
%  \label{cyclic-schedules} %% 1.2.1
%\section{Problem Description.}\label{problemdescription} %% 2.

% Text of your paper here

\section{Literature and Theory Development }
 
\subsection{Literature on Reviews' variations    }
Prior to developing our hypotheses, we will briefly review the current literature on product reviews and their impact.  
The broader issue of product reviews has been addressed in current literature. Following the availability of data,  \cite{chevalier2006effect} found a positive relationship between reviews and product sales in publishing, \cite{wu2015economic} quantified the economic value per piece of positive reviews in helping consumers choosing restaurants, \cite{chintagunta2010effects} extended the study of reviews to movie box office, and \cite{xu2016interplay} furthered our understanding of reviews' role in physician choice. 

Within the literature on the \textbf{variations} of reviews, existing literature has largely studied from the perspective of product and their \textbf{own} reviews' variations. \cite{sun2012does} draw from the consumer utility theory with a Hoteling framework and a DID empirical setting on book sales; they found that higher variance of ratings could increase demand via its signaling effect. \cite{yi2019leveraging} studied the moderating effect of the variance of review... \cite{wang2015user} found that in multiple settings (movie box office, books  digital cameras and lab setting) high user reviews variance can be a double edged sword, and it could be positive if user reviews variance elicit a sense of uniqueness , or negative if it signaled poor quality. \cite{west1998integrating} linked box office response to reviews consensus to consumer aspirations which highlighted opposite effect of reviews variances.  

Another way to understand the impact of reviews on installers activities is from the perspective of performance reviews.  Product ratings is a form of performance reviews on the product or services that sellers/suppliers provided. In the stream of performance reviews literature, \cite{song2017closing} found public performance feedback to encourage low-performing physicians, while \cite{eyring2018performance} found a more nuanced impact of performance reviews in a classroom setting, citing that the reference point matters;  \cite{tan2019you} found a non-linear impact of coworker performance reviews. 
A corollary question to the studies of performance feedback is the uncertainty in feedbacks. \cite{marinovic2015credibility} formulated a game-theoretic model that featured a principal giving noise-mixed performance feedback. \cite{bolton2019inflated} studied inflation in performance feedbacks on the AirBnB setting. This question is related to our study of the variation on reviews. 
 
 
% \begin{APPENDIX}{<Title of the Appendix>}
% \end{APPENDIX}
%
%   or
%
% \begin{APPENDICES}
% \section{<Title of Section A>}
% \section{<Title of Section B>}
% etc
% \end{APPENDICES}


% Acknowledgments here
\ACKNOWLEDGMENT{The authors gratefully acknowledge the existence of
the Journal of Irreproducible Results and the support of the Society
for the Preservation of Inane Research.}


% References here (outcomment the appropriate case)

% CASE 1: BiBTeX used to constantly update the references
%   (while the paper is being written).
\bibliographystyle{informs2014} % outcomment this and next line in Case 1
\bibliography{solarlits} % if more than one, comma separated

% CASE 2: BiBTeX used to generate mypaper.bbl (to be further fine tuned)
%\input{mypaper.bbl} % outcomment this line in Case 2

%If you don't use BiBTex, you can manually itemize references as shown below.

\bibliographystyle{nonumber}

 
%%%%%%%%%%%%%%%%%
\end{document}
%%%%%%%%%%%%%%%%%

