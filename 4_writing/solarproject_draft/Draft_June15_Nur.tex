%%%%%%%%%%%%%%%%%%%%%%%%%%%%%%%%%%%%%%%%%%%%%%%%%%%%%%%%%%%%%%%%%%%%%%%%%%%%
%% Author template for Management Science (mnsc) for articles with e-companion (EC)
%% Mirko Janc, Ph.D., INFORMS, mirko.janc@informs.org
%% ver. 0.95, December 2010
%%%%%%%%%%%%%%%%%%%%%%%%%%%%%%%%%%%%%%%%%%%%%%%%%%%%%%%%%%%%%%%%%%%%%%%%%%%%
%\documentclass[mnsc,blindrev]{informs3} % current default for manuscript submission
%\documentclass[mnsc,nonblindrev]{informs3}
\documentclass[msom,blindrev]{informs3}
\OneAndAHalfSpacedXI % current default line spacing
%%\OneAndAHalfSpacedXII
%%\DoubleSpacedXII
%\DoubleSpacedXI

% If hyperref is used, dvi-to-ps driver of choice must be declared as
%   an additional option to the \documentstyle. For example
%\documentclass[dvips,mnsc]{informs3}      % if dvips is used
%\documentclass[dvipsone,mnsc]{informs3}   % if dvipsone is used, etc.

% Private macros here (check that there is no clash with the style)
\usepackage{graphicx}

% Natbib setup for author-year style
\usepackage{natbib}
\bibpunct[, ]{(}{)}{,}{a}{}{,}%
\def\bibfont{\small}%
\def\bibsep{\smallskipamount}%
\def\bibhang{24pt}%
\def\newblock{\ }%
\def\BIBand{and}%

\usepackage{booktabs}

\usepackage[flushleft]{threeparttable}
\usepackage{float}

%\usepackage{csquotes}
\usepackage[UKenglish,USenglish]{babel}
%%package to comment a whole block
\usepackage{verbatim}
%% Setup of theorem styles. Outcomment only one.
%% Preferred default is the first option.
\TheoremsNumberedThrough     % Preferred (Theorem 1, Lemma 1, Theorem 2)
%\TheoremsNumberedByChapter  % (Theorem 1.1, Lema 1.1, Theorem 1.2)
\ECRepeatTheorems

%% Setup of the equation numbering system. Outcomment only one.
%% Preferred default is the first option.
\EquationsNumberedThrough    % Default: (1), (2), ...
%\EquationsNumberedBySection % (1.1), (1.2), ...

% For new submissions, leave this number blank.
% For revisions, input the manuscript number assigned by the on-line
% system along with a suffix ".Rx" where x is the revision number.
\MANUSCRIPTNO{}

%%%%%%%%%%%%%%%%
\begin{document}
	%%%%%%%%%%%%%%%%
	
	% Outcomment only when entries are known. Otherwise leave as is and
	%   default values will be used.
	%\setcounter{page}{1}
	%\VOLUME{00}%
	%\NO{0}%
	%\MONTH{Xxxxx}% (month or a similar seasonal id)
	%\YEAR{0000}% e.g., 2005
	%\FIRSTPAGE{000}%
	%\LASTPAGE{000}%
	%\SHORTYEAR{00}% shortened year (two-digit)
	%\ISSUE{0000} %
	%\LONGFIRSTPAGE{0001} %
	%\DOI{10.1287/xxxx.0000.0000}%
	
	% Author's names for the running heads
	% Sample depending on the number of authors;
	% \RUNAUTHOR{Jones}
	% \RUNAUTHOR{Jones and Wilson}
	% \RUNAUTHOR{Jones, Miller, and Wilson}
	% \RUNAUTHOR{Jones et al.} % for four or more authors
	% Enter authors following the given pattern:
	%\RUNAUTHOR{}
	
	% Title or shortened title suitable for running heads. Sample:
	% \RUNTITLE{Bundling Information Goods of Decreasing Value}
	% Enter the (shortened) title:
	\RUNTITLE{Do Noisy Customer Reviews Discourage Platform Sellers?}
	%INhibitator, Impede, Hinder, Inpe
	% Full title. Sample:
	% \TITLE{Bundling Information Goods of Decreasing Value}
	% Enter the full title:
	\TITLE{Do Noisy Customer Reviews Discourage Platform Sellers? Empirical and Textual Analysis of an Online Solar Marketplace with Deep Learning}
	% Hinder - Noisy Customer Reviews in an Online Solar Marketplace: Discouragement to Platform Sellers?
	
	% Block of authors and their affiliations starts here:
	% NOTE: Authors with same affiliation, if the order of authors allows,
	%   should be entered in ONE field, separated by a comma.
	%   \EMAIL field can be repeated if more than one author
\ARTICLEAUTHORS{%
\AUTHOR{Herbie Huang}
\AFF{Kenan-Flagler Business School, The University of North Caroline at Chapel Hill, NC  27599, \EMAIL{Herbie\_Huang@kenan-flagler.unc.edu}} %, \URL{}}
\AUTHOR{Nur Sunar}
\AFF{Kenan-Flagler Business School, The University of North Caroline at Chapel Hill, NC  27599, \EMAIL{Nur\_Sunar@kenan-flagler.unc.edu}}
\AUTHOR{Jay Swaminathan}
\AFF{Kenan-Flagler Business School, The University of North Caroline at Chapel Hill, NC  27599, \EMAIL{msj@unc.edu}}
% Enter all authors
} % end of the block

\ABSTRACT{
\textbf{Problem definition:} Customer reviews are essential components of online marketplaces.  However, reviews typically vary; ratings of a product or service are rarely all the same. In many service marketplaces, supply-side participants are active. That is, a seller needs to make a proposal to serve each customer. In such marketplaces, it is not clear how (or if) the dispersion in customer reviews affects seller's activity level, i.e., logged number of proposals, and number of matches in the marketplace. Our paper empirically examines this by considering both ratings and text reviews.

\noindent \textbf{Academic/Practical Relevance:}  To our knowledge, this is the first paper that empirically studies how the review dispersion affects a seller's activity level and the number of matches in an online marketplace with active sellers. Distinct from literature, our paper examines the relationship between the review dispersion and \emph{supply-side dynamics} of an online service marketplace.

\noindent \textbf{Methodology:} We collaborated with one of the largest online solar marketplaces in the U.S. that connects potential solar panel adopters with installers. We obtained a unique proprietary data set from the marketplace for 2013 - 2018. We complement this with public data sets. Our analysis uses the advanced deep-learning natural-language-processing model BERT developed by Google AI, a state-of-the-art clustering algorithm and traditional econometrics methods.

\noindent \textbf{Results:}  We find that the dispersion in customer reviews has a significant and inverted U-shaped effect on an installer's activity level in the online marketplace. Specifically, installer's activity level increases with the review dispersion if and only if that dispersion is below a certain threshold. Above that threshold, more dispersion in reviews lowers the installer's activity level in the marketplace. Furthermore, we identify a significant and inverted U-shaped relationship between the market-level review dispersion and transactions.

\noindent\textbf{Managerial Implications:} Our findings provide valuable insights to marketplace operators about the implications of review dispersion for marketplace operations.
}%

%Our results indicate that the average review of a seller or its competitors do not have a consistent significant impact on the seller’s activity level on the platform.  However, we find that the dispersion of the seller’s own reviews and the dispersion of competitors’ reviews both have a significant impact on the seller’s activity level.  Our results show that structurally, both of these dispersion measures affect the activity level of a seller in the same way. Specifically, we identify an inverted-U relationship between the seller’s activity level and each of these dispersion measures. That is, an increase in either dispersion measure increases the seller activity on the platform if and only if the dispersion is low. When the dispersion is high, an increase in either dispersion measure lowers the seller activity on the platform.  We also identify an inverted-U relationship between matching and the review dispersion at a local market level on the platform.
% Sample
%\KEYWORDS{deterministic inventory theory; infinite linear programming duality;
%  existence of optimal policies; semi-Markov decision process; cyclic schedule}

% Fill in data. If unknown, outcomment the field
\KEYWORDS{online marketplace, natural language processing, customer reviews, solar PV installation} \HISTORY{Update: November, 2019}

\maketitle
	%%%%%%%%%%%%%%%%%%%%%%%%%%%%%%%%%%%%%%%%%%%%%%%%%%%%%%%%%%%%%%%%%%%%%%
	
	% Samples of sectioning (and labeling) in MNSC
	% NOTE: (1) \section and \subsection do NOT end with a period
	%       (2) \subsubsection and lower need end punctuation
	%       (3) capitalization is as shown (title style).
	%
	%\section{Introduction.}\label{intro} %%1.
	%\subsection{Duality and the Classical EOQ Problem.}\label{class-EOQ} %% 1.1.
	%\subsection{Outline.}\label{outline1} %% 1.2.
	%\subsubsection{Cyclic Schedules for the General Deterministic SMDP.}
	%  \label{cyclic-schedules} %% 1.2.1
	%\section{Problem Description.}\label{problemdescription} %% 2.
	
	% Text of your paper here
\vspace{-20pt}	
	\section{Introduction}
	
	
	Online marketplaces are reshaping numerous sectors, ranging from retail to clean technology. In 2019, gross merchandise sales of top online marketplaces across the globe exceeded the astonishing \$2 trillion milestone, with a 22\% growth \citep{onlinemarket}. Online customer reviews, which are evaluations of a product or service by former users, are vital for online marketplaces because they are essential in customer shopping experience.  The customer-side impacts of online reviews are evident. According to the \cite{Northwestern}, nearly 95\% of customers read online reviews before making a purchase.
	Customers pay attention to online reviews, and reviews can significantly influence customer perception \citep{askalidis2016value,reviewmatter}. However, very little is known about the supply-side impacts of online reviews. In many service marketplaces, supply-side participants are \emph{active}. That is, a seller needs to make a proposal to serve each customer. In these marketplaces, it is not clear how (or if) online reviews affect supply-side activities and overall market transactions. Thus, for such marketplaces, understanding the impact of customer reviews is of paramount importance to a marketplace operator. This is the main focus of our work.
	
	
	Online reviews can be in a rating or text format. Ratings are typically measured on a 1 to 5 scale, with 1 being poor and 5 being excellent, while text reviews include customer sentiments about the product or service in words.  There is a growing interest in studying customer ratings in various contexts. The vast majority of this literature investigates how \emph{average} customer ratings impact a \emph{single} firm's sales. Focusing on books and movies, several studies conclude that an improvement in a product's average rating increases its sales (e.g., \cite{chintagunta2010effects} and \cite{chevalier2006effect}). Regarding services, \cite{luca2016reviews} finds that the average rating of a restaurant has a positive impact on its revenue. There are also a few studies that show that the average ratings of a product may not have a significant impact on its sales (e.g., \cite{duan2008online}).  In practice, customer ratings typically vary; it is rare to find a product or service whose ratings are all the same. Despite this, surprisingly, the implications of rating dispersion are severely understudied in the literature (see Section \ref{Sec: Lit}). Our paper contributes to the literature by studying how online review \emph{dispersion} impacts  a key behavior of online marketplace sellers and the online marketplace that consists of multiple active sellers. To the best of our knowledge, there is no prior work that investigates this topic.
	
%A distinct feature of our work is the examination of the \emph{supply-side} impact of reviews in an online marketplace.
	
	% 25\% increase in 2018 https://iea-pvps.org/wp-content/uploads/2020/02/5319-iea-pvps-report-2019-08-lr.pdf
	Our paper considers an online solar marketplace as a context. Solar energy is booming in the world, with a dazzling 34\% growth worldwide in 2017 \citep{iea2018snapshot}. In the U.S., the annual generation from solar photovoltaics (PV) increased by nearly a factor of 4 from 2014 to 2019, and is estimated to more than triple from 2019 to 2030 \citep{USEIA-I,USEIA-II}. A key contributor to this growth is increasing solar panel adoption by electricity end-users (e.g., residential customers). By adopting solar panels, electricity end-users generate their own electricity, reducing their reliance on electric utility companies. In the U.S., the residential solar capacity increased by a factor of 4.18 from 2014 to 2019 \citep{USEIA-III}, and is forecasted to grow 25\% per year \citep{weaver_2019,seia,gtmsolar2018}.
	
	Online marketplaces are transforming the rooftop solar panel adoption process across the United States. An online solar marketplace is a digital platform that connects a potential panel adopter with installers, facilitating the adoption process for electricity end-users. Customers are increasingly interested in connecting with rooftop panel installers through online marketplaces. According to a recent report about a leading online solar marketplace, such an interest doubled in 11 major states of the U.S. between 2017 to 2018 \citep{energysageintel19}.  In this paper, we analyze a novel data set we obtained from one of the largest online solar marketplaces in the United States.
	
	
	The online solar marketplace we study has two salient features. First, for every incoming customer, each installer in a certain region decides whether to serve that customer or not. The installer makes a proposal (bid) if it is willing to serve the customer. Only after the installer's proposal, the installer is listed as available for the customer. This is in contrast to online marketplaces such as Amazon where sellers do not bid for a potential customer. Second, the competition among installers is local. That is, only the installers located in a particular geographical area bid for each customer, and this geographical area is not restricted to city or town boundaries. This is different from online marketplaces like Amazon where the competition among sellers occurs at the entire marketplace level. This difference creates a unique challenge, that is, to identify \emph{local markets} for installers. In our study, we overcome this challenge via a state-of-the-art clustering algorithm.
	
	Our paper considers two key metrics: the number of proposals by each installer, which represents the number of customers each installer is willing to serve on the marketplace, and the number of successful proposals - i.e., \emph{matches} - in the marketplace. The former is relevant to the growth prospect of the online marketplace, which is an important measure for investors \citep{baker}.  The latter metric matters as it is commonly used in the financial valuation of online marketplaces \citep{boris_2018,galston_2017}. Hereafter, for brevity, the logged number of proposals by an installer will be referred to as the installer's \emph{activity level}.
	
	Our analysis is centered around the following three main research questions. (i) Does the dispersion in an installer's customer reviews have a significant impact on the installer's activity level in the online marketplace?  If so, what is the direction of the impact? (ii) Does the dispersion in competitors' customer reviews have a significant impact on an installer's activity level in the online marketplace? If so, what is the direction of the impact? (iii) How does the review dispersion impact the number of matches in the marketplace?  In answering these questions, we consider both  ratings and text reviews made by verified buyers. To consider these two formats, in addition to traditional econometrics methods, we employ the BERT technique, which is an advanced natural language processing model implemented by Google in late 2019. To our knowledge, our paper is the first that employs this technique in the OM literature.
	
\subsection{Main Findings and Contributions}
	
Our paper makes four main contributions to the literature. First, to the best of our knowledge, there is no prior work that empirically investigates how the dispersion in customer reviews impacts a firm's activity level (i.e., logged number of proposals) in an online marketplace. Our paper studies this, and shows that the dispersion in an installer's reviews has a significant and inverted U-shaped impact on its activity level in the online marketplace. Thus, a firm's noisy reviews increase the firm's activity level if and only if its review dispersion is lower than a threshold; beyond that threshold, noisy reviews hurt the firm's activity level in the online marketplace.
	
	
Second, to our knowledge, our paper is the first that studies how the dispersion in competitor reviews impacts a firm's activity level in an online marketplace. In this context, we find that competitors' rating dispersion has a significant and inverted U-shaped impact on the installer's activity level. This suggests that a firm's and its competitors' rating dispersions have the same structural impact on the firm's activity level in the online marketplace.

Third, to our knowledge, our paper is the first to empirically analyze how the review dispersion affects the number of matches in an online marketplace where sellers have to make a proposal to win a customer. Regarding this, we identify a significant and inverted U-shaped relationship between the number of matches and the review dispersion at a local market level. This finding has a key implication for an online marketplace operator: Having all sellers with 5 stars might not be favorable to the marketplace operator. Review dispersion up to a particular level can help an online marketplace operator in terms of number of matches.

Fourth, our paper provides a showcase for a state-of-the-art advanced clustering method (OPTICS) and a very recent deep-learning based natural-language-processing model (BERT). These methods have not been used in the OM literature yet, and have the potential to facilitate research in various contexts.

	
	%
	%Our analysis uses the advanced natural language processing model recently developed and implemented by Google AI, a state-of-the-art clustering algorithm and traditional econometrics methods.
	
	%Different from these papers, we took a perspective of the marketplace operator. The marketplace perspective is an important one, especially from the marketplace providers' perspectives. Many new businesses are running a marketplace business model, and have designed the customer ratings functionality an essential part of the platform experience (CITE SOMETHING). In our work, we use the total number of successful proposals on a relevant local market to gauge the health of the marketplace. Total number of success proposals as a performance metric is consistent with common business practices in the investment circle \citep{boris_2018,galston_2017} as it is tied to a marketplace business's valuation. \\
	%
	%Our objective is to understand the impact of review dispersion on the activity level of each participating supplier on the platform, which has not been studied before. Our study provides insights into the operation of a marketplace and ties reviews to marketplace matching metrics.
	%Our paper is relevant to the
	
	%\cite{atasu2020sustainable}
	
\subsection{Related Literature} \label{Sec: Lit}
	
	Our paper contributes to the sustainable operations literature by examining an online marketplace that facilitates solar PV adoption. Here, we will only mention the most relevant papers that includes a data analysis. Interested readers can find an excellent review in \cite{HLee}. In this stream, various papers analytically study solar and wind technologies while calibrating their models with real-life data (see, e.g., \cite{alanwolf}, \cite{NJ}, \cite{SunarandBirge}, and references therein). There are also several papers that empirically study green technologies. These include carbon abatement technologies (e.g., \cite{blanco2020carbon}, \cite{Corbett2}), waste exchanges (e.g., \cite{Suvrat}) and  off-grid lighting solutions (e.g., \cite{uppari}, \cite{Kamalini2019}).
	To the best of our knowledge, there is no prior work that considers customers reviews and an online solar marketplace in this literature.
	
	%Kenneth and Brian Bollinger
	
	
	%Feedback, social comparison, motivation literature
	%we also find clinics in this group exhibit rank response behavior, specifically Last-Place Aversion; in particular, clinics near Last-Place outperform their corresponding controls by 23\%. further intervention, a group which only received performance feedback – in the form of rankings – based on clinic flu shot growth, and a group which only received financial incentives based on clinic flu shot growth. In line with literature on treatment effects, we define “treated clinics,” as the clinics outside the control group who received either performance rankings or financial incentives. We find the introduction of these attention cues (or “treatment”), i.e. financial incentives and relative performance feedback (RPF), led to an increase in flu shots for these treated clinics. We also find that the clinics who receive performance feedback outperform both other groups.
	%Additionally, most RPF studies focus on individual level feedback within the context of a single firm or school; few of these studies examine firm- or team-level feedback (e.g. Delfgaauw et al. 2013, 2015). Among multi-firm field experiments (in-line with our study), only Delfgaauw et al. (2013) can separately compare the effect of incentives versus the effect of feedback, and the authors find no statistical difference between the two treatments.
	%
	%There are a few distinct differences between our paper. First, our analysis considers a multi-firm level setting
	
	
	Our paper also contributes to the literature on online marketplaces. \cite{moreno2014doing} use a transactional data set from an intermediary for software development services. The authors establish that for a seller,  a superior reputation primarily increases its likelihood of winning a business. \cite{bimpikis2019managing} use data from a natural experiment in a liquidation auction on a business-to-business platform, and illustrate that the design of the online platform significantly impacts the platform's revenues.  \cite{li2020higher} analyze data from an online peer-to-peer property-rental platform, and show that the market thickness can decrease the number of transactions on the platform. To our knowledge, in this stream, there is no work that studies how review dispersion impacts a firm's activity level (i.e., logged number of proposals) or the number of matches in an online marketplace with active sellers, which are the topics of our study.
	
	
	There are a few papers that study the impact of rating variability on a firm's sales. However, there is no consensus
	about the impact. \cite{clemons2006online} find a positive correlation between the rating dispersion and craft beer sales to provide support for a hyper-differentiation marketing strategy in the craft beer industry. In contrast, \cite{Zhu} show that the rating variation for less popular online games has a negative impact on sales. \cite{luo2013impact} find that the dispersion of brand ratings can drastically hurt the firm value while \cite{zhang2006tapping} concludes that the rating variation does not play a significant role in movie openings. Our paper differs from these studies in several dimensions. First and most importantly, unlike these papers, our paper takes the perspective of a marketplace operator, and studies how review dispersion impacts a key seller action and the number of matches in the marketplace. This is in contrast to the common focus in these papers, which is to understand customer-side impact of online ratings on a single firm. Second, in our setting, the seller must prepare a proposal to win its customers. Such a setting is key to our analysis and not considered by these studies. Third, we consider an online solar marketplace, which differs from studied contexts in essential ways.
	%Apart from the explained literature on online ratings, there are a few other papers that consider the variability in online ratings in their study. Sun (2012) shows that a book's average rating interacts with the variance of its rating in its effect on book sales. Moreover, Zimmermann et al. (2018) theoretically analyze a model to decompose the variance of product's ratings into ones caused by taste differences and quality differences and their impacts.
	
	
	
	
	Finally, our paper is related to the relative performance feedback (RPF) literature. The vast majority of this literature studies how feedback impacts an \emph{individual worker}'s performance. Performance is context-specific, and hence measured in different ways. For example, in a hospital setting, \cite{song2017closing} measure the physician performance by patients' median length of stay, and find that sharing best practices and public RPF (by making physician performance public) improve the performance of low-performing physicians. There are only a few studies that consider firm-level RPF (e.g., \cite{josse2013}). Among those, \cite{staats} is the only one that considers feedback not tied to financial incentives (e.g., any external prize or penalty as in tournaments). \cite{staats} establish that when clinics are informed about their rankings on the flu shot growth, which is the performance metric of interest in their setting, clinics exhibit last-place aversion behavior. Our paper differs from this literature in multiple ways. First, in these studies, providing feedback refers to disclosing a firm's or a worker's relative performance of interest. However, in our study, customer reviews, which are customer feedbacks, do not disclose  firms' activity levels and number of matches in an online marketplace, which are the performance metrics of interest. Furthermore, we study the impact of customer review dispersion. To our knowledge, there is no prior work that empirically studies how feedback dispersion affects firms' actions and marketplace operator's performance in this stream.
	
\subsection{Organization of the Paper}
	
	The remainder of our paper is organized as follows. Section \ref{Sec: Hypothesis} states our hypotheses and describes several mechanisms by which review dispersion may impact firms' activity levels and the number of matches in the online marketplace. Section \ref{Sec: Data} explains our data and context, and includes preliminary analysis. Section \ref{Sec: Installer-level} examines how installer's and competitors' review dispersions impact the installer's activity level in the online marketplace. Section \ref{Sec: Market-level} studies the relationship between the market-level review dispersion and the number of matches in the online marketplace. Section \ref{Sec: TextMining} employs text-mining techniques to utilize both numerical ratings and text reviews in our analysis. Section \ref{Sec: Robustness}  provides various robustness checks, and shows that our findings are robust.
	
	
	
\section{Hypothesis Development} \label{Sec: Hypothesis}
%As the entreprenur and head of TED Chris Anderson famously put it, "Your brand isn’t what you say it is, it’s what Google says it is".
% CITE: https://www.forbes.com/sites/shamahyder/2018/07/18/protect-your-reputation-and-grow-your-brand-using-the-power-of-online-reviews/#2d6ca56328ff
%In practice, advisor to small businesses have warned that if the underlying problem is not addressed, "Should you fail to acknowledge what is being mentioned in the negative review,....efforts to collect more online reviews will simply result in you collecting more negative review" Cite(Neil Patel https://neilpatel.com/blog/how-to-get-more-online-reviews-and-deal-with-bad-ones/ ) .

	Online reviews play many roles in an online marketplace. One role is related to the brand image of firms. Brand image is defined as ``mental construct developed by the consumer on the basis of a few selected impressions among the flood of total impressions'' \citep{reynolds}. Online reviews have become an integral part of the brand image in the age of e-commerce and digitalization \citep{chakraborty2018credibility,chakraborty2018effects,brandimage}. At the same time, online reviews are also seen as a reflection of consumer taste in the market \citep{clemons2006online}. Thus, a firm's review dispersion can have an intricate impact on its marketplace activity level via different mechanisms, which we will explain below.


	On one hand,  a higher review dispersion of an installer may encourage the installer and thus increase its activity level in the online marketplace.  The dispersion in the installer's online reviews can lead to the dispersion in brand image, and that can drastically hurt the firm value \citep{luo2013impact}. When faced with a large review dispersion, an installer may be willing to serve more customers to increase the number of its reviews, thereby reducing its review dispersion and improving the consistency of its brand image. As a result, installers with a higher review dispersion may make more proposals to win more customers in the online marketplace.


%we shall observe installers with a higher review dispersion make more proposals in order to win more customers.

	On the other hand, a higher review dispersion of an installer may also discourage the installer and decrease its activity level in the online marketplace. A higher review dispersion may imply a higher differentiation of customer taste in the market \citep{clemons2006online}. In such a market, making more proposals may impose reputational risks to the installer, potentially due to negative customers or additional polarized reviews. In fact, practitioners warn about these reputational risks, and advise that in the presence of negative customers, efforts to collect more online reviews may results in even more negative reviews for the firm (see, e.g., \citep{Neil}). When faced with reputational risks, firms can be more selective in their project choices to avoid such reputational risks \citep{demirag2011risks,hirshleifer1992managerial}. Thus, as the installer's review dispersion increases, the installer may reduce its activity level to be more selective about which customer to serve in the marketplace.

A higher review dispersion of an installer may also reduce its activity level in the online marketplace through another mechanism. That is, an installer might think that with a higher review dispersion, it is less likely to win a customer compared to its competitors because a higher review dispersion may damage customer perception \citep{Zhu}. Since making a proposal is costly for an installer, when faced with a lower likelihood of winning customer, the installer may reduce its activity level in the marketplace to avoid expenses that do not generate revenues.

 Based on all of these conflicting perspectives, we have the following competing hypotheses:

	
	\noindent\textbf{Hypothesis 1A:} \emph{An increase in an installer's review dispersion increases its activity level in the online marketplace.}
	
	\noindent\textbf{Hypothesis 1B:} \emph{An increase in an installer's review dispersion decreases its activity level in the online marketplace.}


%Reputation and Its Risks https://hbr.org/2007/02/reputation-and-its-risks

The dispersion in competitor reviews may also impact an installer's activity level. Competitors, as will be explained in detail, is a set of installers who share the same local market and potential customers. Similar to our earlier discussion, an increase in the dispersion of competitor reviews may be perceived as a signal of a more polarized market. Seeing competitors receive brand-image-damaging reviews can be perceived as a negative about the market. Thus, when competitor reviews become more disperse, to avoid any reputational harm due to polarized reviews or negative customers, the firm may be more conservative in making proposals. That would reduce the installer's activity level in the online marketplace.
%
%On the other hand, a higher dispersion in competitor reviews might also signal that the online review system is well-functioning and trustworthy. Too many firms with no (or negligibly small) review dispersion might hurt the credibility of the online review system in the marketplace (Maslowska, E., Malthouse, E.C., and Bernritter, S. (2017), “Too Good to be True:
%The Role of Online Reviews’ Features in Probability to Buy.” International Journal of
%Advertising, 36(1) 142-163.). In fact, some platforms
%
% and able to differentiate firms. Too many perfect reviews could seem fake (cite $techcrunch.com/2016/01/03/you-dont-want-a-5-star-review/$). The 5-star rating system could be prone to 'grade-inflation' and have been lamented for lacking the ability to differentiate between good and great providers on multiple platforms such as 99designs, Fiverr, and Upwork (cite $hbr.org/2019/07/the-problems-with-5-star-rating-systems-and-how-to-fix-them$). A marketplace with a well-functioning rating system would be more encouraging to participants, thus promoting a higher activity level.  \\


At the same time, a higher dispersion in competitor reviews can also increase the installer's activity level in the marketplace through another mechanism. A higher dispersion in competitor reviews might hurt the competitor's brand image, and may negatively impact the customers' perception about the competitors' service \citep{chakraborty2018credibility,chakraborty2018effects,Zhu}. This may increase the installer's likelihood of winning any customer compared to its competitors if the installer makes a proposal \citep{demirag2011risks,moreno2014doing}. Given the higher likelihood of winning, it may be favorable for the installer to  bid for more customers so as to improve its sales. That would increase the installer's activity level in the marketplace. In light of these, we have two competing hypotheses:





%	The dispersion in competitor reviews may also impact an installer's activity level. Similar to our earlier discussion, an increase in the dispersion of competitor reviews may be perceived as a signal of a more polarized market. Thus, when competitor reviews become more disperse, to avoid any reputational harm due to polarized reviews or picky customers, the firm may be more conservative in making proposals. That would reduce the installer's activity level in the online marketplace.
%
%
%On the other hand, a higher dispersion in competitor reviews might hurt the competitor's brand image, and may negatively impact the customers' perception about the competitors' service \citep{chakraborty2018credibility,chakraborty2018effects,Zhu}. This may increase the installer's likelihood of winning any customer compared to its competitors if the installer makes a proposal \citep{demirag2011risks,moreno2014doing}. Given the higher likelihood of winning, it may be favorable for the installer to bid for more customers so as to improve its sales. In light of these, we have two competing hypotheses:
%	%(Auction setting de quality artarsa winning probability artar diye bir makale bul)
	
	
	\noindent\textbf{Hypothesis 2A:} \emph{An increase in competitors' review dispersion increases the installer's activity level in the online marketplace.}
	
	\noindent\textbf{Hypothesis 2B:} \emph{An increase in competitors' review dispersion decreases the installer's activity level in the online marketplace.}


Apart from the impact on the installer behavior discussed above, the review dispersion can also influence the number of transactions in the online marketplace.  A match, i.e., an agreement between a customer and any installer, occurs only when the customer is willing to accept an available proposal. A key determinant of a customer's willingness to accept any proposal is the number of proposals she receives from installers, which is a direct consequence of installer activity levels. As explained above, installer activity levels can be impacted by review dispersion via various mechanisms.

By the discussions about Hypotheses 1A, 1B, 2A and 2B, it is not clear how the market-level review dispersion impacts the average number of proposals per customer. There are also conflicting perspectives about how the number of available options affects the customer's willingness to accept any option. On one hand, receiving more proposals might overload the customer, and can decrease the customer's motivation to accept any proposal \citep{scheibehenne2010can,iyengar2000choice}. On the other hand, having more proposals may also increase the customer's motivation to accept a proposal because in a larger set of options, the customer  might be more likely to find a proposal that better matches to her objective \citep{scheibehenne2010can,baumol1956variety}. Thus, an increase in the average number of proposals per customer may increase or decrease the customer's willingness to accept any proposal, i.e., the likelihood of a match. Combining all, an increase in market-level review dispersion can increase or decrease the (logged) number of matches in the online marketplace. Thus, we have the following two competing hypotheses:

%The review dispersion may also increase the market transaction through a different mechanism. Increase in review dispersion might signal that the review system is well-functioning and trustworthy. This is because not having enough review dispersion could make the platform and reviews appear fake if they are all 5-star reviews (cite the 5-star appear fake :  $techcrunch.com/2016/01/03/you-dont-want-a-5-star-review/$)) or low quality if they are all bad reviews. A lack of reviews dispersion would make it hard for consumers to choose (cite the differentiate good to great cite $hbr.org/2019/07/the-problems-with-5-star-rating-systems-and-how-to-fix-them$).  \\
	
	
	\noindent\textbf{Hypothesis 3A:} \emph{An increase in the market-level review dispersion increases the (logged) number of matches in the online marketplace.}
	
	\noindent\textbf{Hypothesis 3B:} \emph{An increase in the market-level review dispersion decreases the (logged) number of matches in the online marketplace.}
	
	
	
	%
	%Our paper is also related to papers that investigates the effect of ratings on a single firm's performance metrics. In that stream, there is no consensus about the ultimate impact of dispersion of ratings on the firm's performance metric. Studies have demonstrated positive impacts \citep{chintagunta2010effects,chevalier2006effect,dellarocas2007exploring}, insignificant impact \citep{duan2008online}, and negative impacts in some instances \citep{wang2015user}.
	%
	%In the literature, there are papers that show the positive impact of reviews on sales. There are also other papers that demonstrate  (AVERAGE LIT (Literature considered average effect)).\\
	
	%H1: Higher reviews dispersion is associated with higher levels of installer activities.
	%Installers could be encouraged by the differentiating aspect of reviews and view the marketplace reviews system working as designed.
	%H2: Higher reviews dispersion is associated with lower levels of installer activities.
	%Installers could be concerned about picky customers who may damage their reputation.
	%H3: Higher average review is associated with higher levels of installer activities
	%Installer could be encouraged by good reviews and use them to attract more customers.
	%H4: Higher average review is associated with lower levels of installer activities
	%Installers could be more confident about their chances (of proposals being accepted) and act more conservatively at reaching out.
	%
	%
	%
	%
	%
	%---
	%Impact of reputation on firm behavior
	%
	%
	%The importance of reputation effects in competitive markets was pointed out by
	%Friedman (1962) and Akerlof (1970), who predict that firms with good reputations
	%will expand and gain market share. This is indeed seen in many markets for consumer
	%products, but educational markets often display a different pattern. Bound,
	%Herschbein, and Long (2009) document that while the number of college applicants
	%nearly doubled since the 1970s, elite colleges essentially did not grow but rather
	%became increasingly selective.
	%
	%
	%Friedman, Milton. 1962. Capitalism and Freedom. Chicago: University of Chicago Press.
	%
	%Akerlof, George A. 1970. “The Market for ‘Lemons’: Quality Uncertainty and the Market Mechanism.”
	%Quarterly Journal of Economics 84 (3): 488–500.
	%
	%Bound, John, Brad Hershbein, and Bridget Terry Long. 2009. “Playing the Admissions Game: Student
	%Reactions to Increasing College Competition.” Journal of Economic Perspectives 23 (4): 119–46.
	%
	%Shouldice education system
	%
	%Managerial Conservatism, Project Choice, and Debt
	%David Hirshleifer,  Anjan V Thakor
	%The Review of Financial Studies, Volume 5, Issue 3, July 1992, Pages 437–470, https://doi.org/10.1093/rfs/5.3.437
	%
	%----
	%This relationship matters because there are various studies that show that a customer's willingness to accept any offer depends on the number of available offers.  On one hand, having more options might overload the customer, and can decrease the customer's motivation to accept any offer \cigscheibehenne2010can. On the other hand, having more options may motivate the customer to accept an offer as s/he  might be more likely to find an offer that matches better
	%
	%the average number of installer proposals per customer. Furthermore,
	%there are different perspectives about the relationship between number of options and customer
	%
	%
	%On one hand, the aggregate effect of Hypotheses 1a, 1b, 2a and 2b may be increased number of proposals by firms in a market3. Increased number of proposals per customer may reduce customer's willingness to accept any offer. This is because having too many options might be overload a customer, and can decrease their willingness to make any choice. \textbf{(Can There Ever Be Too Many Options? A Meta-Analytic Review of Choice Overload, Benjamin Scheibehenne,  Rainer Greifeneder,  Peter M. Todd, Journal of Consumer Research, Volume 37, Issue 3, October 2010, Pages 409–425) - Iyengar, Sheena S., \& Lepper, Mark R. (2000). When choice is demotivating:
	%Can one desire too much of a good thing? Journal of Personality and Social
	%Psychology, 79(December), 995–1006. } If the decrease in customers' willingness to accept any offers surpass the
	%The choice-overload hypothesis states that increasing the number of alternatives reduces people's motivation to choose
	%Difficulty of making a decision when there are too many options. On the other hand, several studies show the positive impact of larger sets an increase in the average number of installer proposals can give more flexibility to customers
	%
	%The most intuitive benefit, featured prominently in
	%economics research, is that the greater the number of options in the
	%choice set, the higher the likelihood that consumers can find a close
	%match to their purchase goals (Baumol \& Ide, 1956; Hotelling,
	%1929).
	%
	%
	%Baumol, William J., & Ide, Edward A. (1956). Variety in retailing.
	%Management Science, 3(October), 93–101.
	%
	%Hotelling, Harold (1929). Stability in competition. The Economic Journal,
	%39(March), 41–57.
	
	
	
	
	
	
	
	%
	%\subsection{How Reviews Dispersion Impacts Activity Intensity(Literature Review) }
	% In this section we describe several mechanisms by which reviews dispersion may impact installers activity intensity on a platform. \\
	% Previous studies have established the important of performance feedback on worker productivity. In a hospital setting \cite{song2017closing} found a positive impact from public performance feedback to low-performing physicians. In a restaurant setting, coworker performances influence waiters own `up-selling' behavior, a reflection of efforts, in an non-linear, inverse U-shape fashion. \\
	%The concept of \textbf{ratings dispersion} has been explored in marketing literature. For example, \cite{luo2013impact} examined the brand ratings dispersion and its impact on firm values. In the economics literature, \cite{marinovic2015credibility} modeled the phenomenon of performance feedback signal with a noise in a principal-agent model and illustrated feedback noise has potential of inducing agents efforts. Overall, the impact of feedbacks dispersion is less explored in an operations setting. \\
	%The impact of high ratings could be two-folded. On the one hand, high variations could be an indicator that the ratings scheme is functioning as it is designed - it rewards good installer and records the bad deeds of the bad ones. It could encourage installers to pursue more leads in order to get a chance to be evaluated.
	%On the other hand, a high ratings variation could also be taken as a sign of picky customers on the market. Installers fear of establishing bad permanent reputation will be more cautious when getting into a market of potentially picky customers. \\
	%In this study, we make use of the detailed installer level activities data. We explore not only the impact of ratings, but more importantly, the nuanced impact of the ratings dispersion and reviews variation.
	
	
	
	
\section{Data and Setting} \label{Sec: Data}
	
	For our study, we collaborated with one of the largest online solar marketplaces in the U.S., and obtained proprietary marketplace data from the company. We also complement this data set with Tracking The Sun (TTS) data set from the Lawrence Berkeley National Laboratory. TTS is a comprehensive data set on U.S. solar panel installations. Below, we will  provide further details about our data and the setting of the online solar marketplace we study.
	
	
	
	
	
	%We use a compilation of proprietary and publicly available data about residential solar markets. The focus of the study is about the actions and outcomes of an online marketplace for residential solar installations. We obtained, via collaboration with the marketplace company, the full record of customer reviews and installer actions on a monthly level from 2013 to 2018. We complement the marketplace data with Tracking The Sun (TTS) data set from Lawrence Berkeley National Laboratory. TTS aggregates data from more than 60 state and utility incentive programs. The full TTS data set covers more than 80\% of the U.S. PV
	%market, making it the most comprehensive extant U.S. PV data set. It contains installation level information such as installer name, unit size and price which allow us to construct a big picture of solar installation activities that are happening on and off the marketplace.
	
	\subsection{Online Solar Marketplace}
	
	The solar marketplace we study is an independent shopping website for electricity end-users (e.g., homeowners) who are interested in adopting solar panels.  The marketplace operates in 33 states of the U.S., and allows solar panel installers to maintain a profile, receive information and connect with potential customers in their service areas.
	
	The marketplace operates as follows. First, each customer visits the marketplace website and enters her information, such as the location of her property. Next, installers are informed about the customer's arrival along with her information.  Each installer only serves to a particular region. If the customer's location falls into an installer's service area, the installer decides whether to make a proposal to the customer or not. After the customer observes installer proposals she receives, there are two possible outcomes: Either the customer agrees to work with an installer, i.e., there is a successful \emph{match}, or the customer gives up the process, i.e., there is no matching. If the customer ends up working with the installer, she can leave a review that contains text and a rating ranging from 1 to 5 stars. The marketplace verifies customers who leave reviews. Hence, reviews are considered as authentic and not manipulated.


%Figure \ref{reviews_example} provides an example of how customer reviews are displayed in the online marketplace.
%	
%	
%	\begin{figure}
%		\centering
%		\includegraphics[width=0.81\linewidth]{reviews_example.png}
%		\caption{A sample customer review in the marketplace.}
%		\label{reviews_example}
%	\end{figure}
	
	As a result, the key decision for each installer in the marketplace is whether to make a proposal for a potential customer or not. In light of this, we study how the dispersion in customer reviews impacts (i) an installer's \emph{activity level} in the marketplace, which is a logarithmic transformation of the number of proposals the installer makes in a month, and (ii) the number of monthly matches in the marketplace. For the reasons explained earlier, both of these are important metrics for the marketplace operator.
	
	%from their profiles
	
	
	To investigate (i) and (ii), we obtained a unique panel data set from the online solar marketplace. Our data set contains the data of the marketplace's all vetted installers across the U.S. and full record of their customer reviews from January 2013 to April 2018. There are 416 installers in the marketplace, and we have each installer's monthly activities, i.e., the number of proposals made and the number of proposals won by each installer in every month, during the aforementioned time frame. Each review has a rating, text content, time stamp, and the installer ID and name with which the review is associated. We also have the location information of each installer, as illustrated in Figure \ref{fig: nationalinstallers}. In the marketplace we study, there is no ``closing off'' or explicit exit behavior as in physical retail stores. If an installer prefers to quit the online marketplace, the installer simply becomes inactive, making no proposals to potential customers. Our analysis accounts for such behavior.
	
	
	
	%416 installers
	% installers' monthly activities and all customer reviews from January 2013, which is the beginning of the marketplace, up to April 2018. Specifically, in our data set, there are 416 installers, each installer's monthly activities, i.e., the number of proposals made and the number of proposals won by each installer in every month,  and
	%
	%we have observations about 416 installers about their monthly activities, that is, the number of proposals made and the number of proposals won by each installer in every month, and each installer's all customer reviews.
	
	
	%% Please add the following required packages to your document preamble:
	%% \usepackage{booktabs}
	%\begin{table}[]
	%\centering
	%\begin{tabular}{@{}ll@{}}
	%\toprule
	%           & Description           \\ \midrule
	%Installers & 416 Unique Installers \\
	%Ratings and reviews & 3607 pieces of review records with the rating, text content,timestamp \\
	%Time span  & from 2013 to 2018     \\
	%Monthly Records & 6522 pieces \\
	%Supplementary & Tracking the Sun data \\
	%\bottomrule
	%\end{tabular}
	%\caption{Main Data Source}
	%\label{brief_data_desc}
	%\end{table}
	
	\begin{figure}
		\centering
		\includegraphics[width=1\linewidth]{national_installers.png}
		\caption{Installers in our data set}
		\label{fig: nationalinstallers}
	\end{figure}
	
	
	%We learned from our conversions with practitioners that the online marketplace actively reaches out to solar installers to encourage them to join the platform and help them set up their profiles. So, unlike starting a physical business, installers' fixed cost of entry to the online marketplace is negligibly small (if not zero). This is indeed the case in many other platforms \citep{haddad2015consumer}. Because there is almost no barrier for entry to the online marketplace, we study installers' activity levels after they establish their profiles on the marketplace; we do not explicitly model installers' entry decisions in the marketplace.
	
	
	%Lastly, we don't observe quitting the platform the same way as physical store closes off. We simply observe inactive profiles. Thus we do not explicitly investigate the exit behavior.
	
	
	\subsection{Defining Local Market}
	\label{defining_local_market}
	
	
	Solar panel installation is a combination of product and service. As part of service, installers typically visit customer site multiple times. Thus, each installer only operates  within a certain geographical area, and installers compete ``locally.'' That is, they only compete with installers that are relatively nearby. The caveat is that there are no available data on the installers' service areas. To capture this practical element, we identify ``\emph{local markets}'' within the marketplace so that only installers in the same local market compete with each other.
	
	To geographically segment the marketplace into local markets, we divide installers into multiple \emph{clusters} and treat each cluster as a separate local market. Boundaries of local markets cannot be simply defined by city, county, or congressional district borders because it is common for installers to cross these artificial borders to serve customers. Instead, we use installer locations and the state-of-the-art advanced clustering algorithm called OPTICS (\textit{Ordering Points To Identify the Clustering Structure}) to identify local markets.
	
	The OPTICS routine is an unsupervised machine learning algorithm that identifies density-based clusters in spatial data. It is considered to be an extension of various commonly-used advanced clustering algorithms, such as DBSCAN \citep{kanagala2016comparative}. Among others, an important advantage of the OPTICS algorithm is that it does not require fixing the number of clusters before running the algorithm as in $k$-means clustering method; rather, it identifies the optimal number of clusters using data. Because of this and many other advantages, OPTICS has been applied in various contexts, ranging from political science \citep{davidson2019neighborhood} to geography \citep{teimouri2016method}. To the best of our knowledge, our paper is the first that uses this advanced clustering technique in the OM literature.
	
	%data mining \citep{breunig2000fast},
	
	In light of these, we create the geographic division of local markets with the following steps. First, we collected the 5-digit zipcode of
	every installer in the marketplace. Figure \ref{fig: nationalinstallers} displays the location of every installer in our data set. We then converted each zipcode  to the representative coordinates based on the data provided by the \citet{us_census_bureau_2019}. This transformation is necessary to run the OPTICS algorithm on the location data. The OPTICS algorithm uses the maximum distance between two samples in a cluster as an input variable. Based on our conversations with practitioners, we learned that the majority of customers get a quote from an installer within 90 miles distance of their property. Consistent with this, we used 90 miles as the maximum distance input parameter, and the OPTICS algorithm generated 36 clusters. Each of these clusters geographically defines a local market boundary. Figure \ref{fig: markets} illustrates the centroid of each of these 36 clusters, which represents the centroid of each local market. Hereafter, for brevity, we refer to local markets simply as ``\emph{markets}.''
	
	%\footnote{We also checked the robustness of our results by taking the maximum distance parameter as 100 miles in the OPTICS algorithm. Our insights remain to be valid with that alternative parameter.}
	
	
	
	%\paragraph{OPTICS}  The OPTICS algorithm, short for \textit{Ordering Points To Identify the Clustering Structure}, is what we use to cluster installers' coordinates. The OPTICS routine is completed with the following parameter considerations:  \\
	%\textbf{min samples: }
	%The number of samples in a neighborhood for a point to be considered as a core point.  We use 2 as the default value.
	%\textbf{metric}: haversine distance. Although not ideal, it better reflected the distance between two Latitude/Longitude points and is still fast enough in the clustering algorithm.  \textbf{max eps:} The maximum distance between two samples for one to be considered as in the neighborhood of the other. According to the survey that the marketplace conducted, 90.6 percent of customers get a quote from an installer within 100 miles of their property ( 81.7 percent from 50 miles). \citep{marsh_2019} We used 90 as the parameter to balance between having enough clusters to  make use of the inherent variations and make sure each cluster captured the local market condition. We use this cluster to define our market boundary geographically. Figure \ref{fig:markets} illustrates the centroid of each of these 36 clusters, which represents the centroid of each local market. Hereafter, for brevity, we refer to local markets simply as ``\emph{markets}.''
	
	
	
	%(PROVIDE A PICTURE TO ILLUSTRATE THE CALINSKI-HARABASZ CURVE VS PARAMETER , refer to figure \ref{optics_parameter_gridsearch} and Calinski-Harabasz criteria : \citep{calinski1974dendrite}.
	
	
	
	
	
	%\begin{figure}
	%	\centering
	%	\includegraphics[width=1\linewidth]{histogram_ind_max_reviews_ct.png}
	%	\caption{Reviews Histogram by Installer}
	%	\label{histogram_ind_max_reviews_ct}
	%\end{figure}
	%
	%% Please add the following required packages to your document preamble:
% \usepackage{booktabs}
\begin{table}
\centering
\begin{tabular}{@{}ccccc@{}}
\toprule
 & Unique  & Total Number & Total Number     & Total Number   \\
State    & Installers & of Reviews   & of Installations & of Quotes \\ \midrule  
CO & 13 & 799  & 168  & 13276 \\
MD & 10 & 895  & 144  & 4054  \\
WA & 9  & 902  & 35   & 1266  \\
TX & 27 & 987  & 90   & 12557 \\
FL & 21 & 994  & 141  & 9047  \\
CT & 10 & 1037 & 78   & 2746  \\
NC & 16 & 1100 & 95   & 7066  \\
NJ & 26 & 1674 & 223  & 8215  \\
NY & 32 & 2790 & 265  & 15128 \\
MA & 36 & 3519 & 507  & 19028 \\
CA & 95 & 7703 & 1472 & 98597 \\ \bottomrule
\end{tabular}
\caption{Top 10 States }
\label{summarystats_top10states}
\end{table}
	%
	%
	%
	\begin{figure}
		\centering
		\includegraphics[width=1\linewidth]{markets.jpg}
		\caption{Local Market Centroids}
		\label{fig: markets}
	\end{figure}
	
	\subsection{Measuring Dispersion in Customer Ratings} \label{Subsec: Measure Dispersion}
	
	A key explanatory variable in our regression is the dispersion in customer ratings. This section explains how we measure the rating dispersion. Later, we will also study an extended model by adding the text-based review dispersion as a separate variable in our analysis.  Section \ref{Sec: TextMining} will explain the state-of-the-art natural language processing model we use to measure the review dispersion based on text data.
	
	%\subsubsection{Measuring Rating-Based Dispersion} \label{Subsec: Define Ent}
	
	We measure the rating dispersion by calculating the \emph{entropy} of ratings. In information theory, the entropy is a common way to measure the information content or dispersion (or uncertainty) in a variable's possible realizations. In our setting, because the marketplace has a 5-star rating system, the entropy of ratings is
	\begin{equation}\label{def: entropy}
	H(R)= \sum_{j=1}^{5} \text{Prob}(\text{Rating}=j) \log(1/\text{Prob}(\text{Rating}=j)).
	\end{equation}
	For example, for a set of 5 reviews each with 4 stars (out of 5 stars), the entropy of ratings $\{4,4,4,4,4\}$ is zero. Alternatively, for a set of 5 reviews with ratings $\{3,5,3,5,4\}$, the entropy of ratings is 1.0549. Although both sets have the same average rating of 4, the latter set of ratings provides more information with a higher dispersion, hence has a higher entropy.
	
	
	In light of this, we create three variables that measure the rating entropy in different dimensions for each month $t$. First variable is $\text{Rating\_Entropy\_Self}_{i,t}$, which is the demeaned entropy of ratings associated with installer $i$ up to and including month $t$. Recalling the market defined in Section \ref{defining_local_market}, the second variable is $\text{Rating\_Entropy\_Others}_{i,t}$ that is the demeaned rating entropy of all other installers in installer $i$'s market, up to and and including month $t$. Our third variable is $\text{Rating\_Entropy\_Mkt}_{m,t}$ that represents the demeaned entropy of all ratings in the market $m$, up to and including month $t$. Note that these three variables are centered around their means. This is a standard procedure in settings like ours where the regression includes both linear and quadratic terms of an explanatory variable (see, e.g., \cite{tan2014does}). We also checked the robustness of our findings by replacing these variables with their non-demeaned versions in all our econometric analysis, and we find that all of our insights remain the same with non-demeaned variables.
	
	We measure the rating dispersion by calculating entropy rather than variance of ratings. The reason is two folds: First, the entropy measure provides a higher precision for our data than the variance. That is, two installers with very small difference in rating variance tend to show a larger difference in rating
entropy. Second, when data display multi-modality as our rating data do, entropy is considered to be a better measure than the variance in capturing the dispersion in data \citep{smaldino2013measures}.
	
	
	%********** COPY PASTE ICIN******
	%
	%We transformed $\text{Installer\_Activity}$ and $\text{Market\_Transaction}$ into their natural logarithm. Natural log transformation on continuous dependent variable is a commonly used technique that is embraced by similar work such as \citep{tan2019you}. The distribution of the $\text{Installer\_Activity}$ and $\text{Market\_Transaction}$ are right-skewed with a skewness of $6.9$ and $6.1$ respectively. We transform the data to better conform to normality in order to improve the validity of inference.  As a robustness check, we performed analysis without log transformation and the results are consistent.
	
	
	
	
	\section{Installer-Level Analysis \& Results} \label{Sec: Installer-level}
	
	This section examines the following questions: (i) How does the dispersion in an installer's ratings affect its \emph{activity level}, which is the logged  number of proposals generated by the installer? (ii) How does the dispersion in competitors' ratings impact the installer's activity level? By studying these questions, we test Hypotheses 1A, 1B, 2A and 2B in Section \ref{Sec: Hypothesis}.
	
	
	We will only use numerical ratings in this section. Later, Section \ref{Sec: TextMining}  will extend our analysis to include text reviews. Section \ref{Sec: Robustness} will check the robustness of our findings in various dimensions, and address potential endogeneity concerns in an extended model in Section \ref{Sec: Dynamic Panel}.
	
	\subsection{Regression Equation \& Controls}
	
	To answer (i) and (ii) above, we consider a regression model where the dependent variable is a natural logarithmic transformation of the number of proposals made by an installer. Formally, indexing installers, months and markets by $i$, $t$ and $m$, respectively, the dependent variable in our regression is  $\text{Installer\_Activity}_{i,m,t+1}$, which is equal to $\ln$(1 $+$ number of proposals generated by installer $i$) in the market $m$ during month $t$+1. We make this transformation because the number of installer proposals has a right-skewed distribution, and log transformation increases the normality of errors, thereby further improves the validity of inference. This is a standard procedure in the literature (see, e.g., \citet{song2017closing,tan2014does}, among others). As a robustness check, we also performed the analysis without a logarithmic transformation and found that results are consistent.
	
	Two of our key explanatory  variables are $\text{Rating\_Entropy\_Self}_{i,t}$ and  $\text{Rating\_Entropy\_Others}_{i,t}$, which are defined in Section \ref{Subsec: Measure Dispersion}. Because we have competing hypotheses, we also allow for nonlinear relationships between each of these explanatory variables and the dependent variable by including explanatory variables $\text{Rating\_Entropy\_Self}_{i,t}^{2}$ and $\text{Rating\_Entropy\_Others}_{i,t}^{2}$ in our regression. Specifically, our regression equation is as follows:
	\begin{align}  \nonumber
	\text{Installer\_Activity}_{i,m,t+1}=&\beta_{0}+\beta_{1} \text{Rating\_Entropy\_Self}_{i,t}+\beta_{2} \text{Rating\_Entropy\_Self}_{i,t}^ {2}
	\\ \nonumber
	&+\beta_{3} \text{Rating\_Entropy\_Others}_{i,t}  +\beta_{4}\text{Rating\_Entropy\_Others}_{i,t}^{2} \\ \label{model_ind_3}
	&+ \text{Controls}_{i,m,t}+ \alpha_{i} + \epsilon_{i,t+1}.
	\end{align}
	Here, $\epsilon$ is the installer-level error term, and represents random factors that are unobservable in the data and affect the installer activity.  We run two versions of \eqref{model_ind_3}: In one version, we consider $\alpha_{i}$ as a fixed effect whereas in the alternative version, we consider it as a random effect. To determine which model is more appropriate for our data, we run the Durbin-Wu-Hausman test where the null hypothesis is that the random-effect model is preferred while the alternative is the fixed-effect model. With a p-value $<0.0001$, we reject the null hypothesis and conclude that the fixed-effect model is more appropriate. We also establish the significance of the fixed effect in \eqref{model_ind_3} with the $F$-test ($\chi^{2}(13)=44.23, p < 0.0001$). Thus, we focus on \eqref{model_ind_3} with the installer-level fixed effect $\alpha_{i}$ that controls for time-invariant characteristics of each installer.
	
	In this regression model, the key coefficients of interests are $\beta_{1}$, $\beta_{2}$, $\beta_{3}$ and $\beta_{4}$. The values of these coefficients together with the significance of the associated variables will uncover how the rating entropy impacts the installer's activity level in the online marketplace.
	
	
	The regression \eqref{model_ind_3} includes various installer-level or market-level control variables ($\text{Controls}_{i,m,t}$).  To account for the state-level renewable policy effects, we include state dummies as control variables. We have 33 such variables. We account for the impact of the solar panel prices on installers' activity levels by considering $\text{Price\_Difference}_{i,t}$ as another control variable.  In practice, because solar PV systems vary in size, price per KW is a common way to represent the price of the installed solar panel. We collected each installer's price for 1 KW solar panel by matching names and zipcodes and using the TTS data set. Based on this, we compute the variable $\text{Price\_Difference}_{i,t}$ by taking the logarithm of the difference between installer $i$'s price and the average price of its competitors that operate in the same market in month $t$.  We control for the average rating of each installer $i$ as well as the average rating of its competitors in the market for month $t$ by including variables $\text{Average\_Rating\_Self}_{i,t}$ and $\text{Average\_Rating\_Others}_{i,t}$ in \eqref{model_ind_3}. We also control for the installer's experience by including the variable $\text{Experience}_{i,t}$ that is the logged number of years the installer has been installing solar systems up to (and including) month $t$. We collected this information one by one from each installer's website. Another control variable in \eqref{model_ind_3} is $\text{Market\_LogRevenue}_{m,t}$  that measures the logged total dollar value of all solar installations within market $m$ during month $t$. To create this variable, we augment the market boundaries identified in Section \ref{defining_local_market} with the TTS data. This variable aims to capture total solar installations opportunities in the market, and can be seen as a proxy for the favorableness of the solar installation market. As the final control variable, we consider  $\text{Review\_Counts}_{i,t}$ which is the number of each installer $i$'s reviews up to (and including) month $t$.
	
	
	
	
	
	% the following regression equation is used to estimate the impact of ratings dispersion (own ratings dispersion: $Ent_{i,self,t}$ ; others ratings dispersion: $Ent_{i,others,t}$) on focal installer's activity intensities $ActInt_{i,m,t}$:
	%\begin{equation}
	%    ActInt_{i,m,t+1}=\beta_{0}+\beta_{11} Ent_{i,m,others,t}+\beta_{2}Ent_{i,m,others,t}^2+
	%   Controls_{i,t}+\alpha_{i}+\epsilon_{i,m,t}
	%   \label{model_ind_1}
	%\end{equation}
	%
	%\begin{equation}
	%    ActInt_{i,m,t+1}=\beta_{3}+\beta_{4} Ent_{i,self,t}+\beta_{5}Ent_{i,self,t}^2+
	%   Controls_{it}+\alpha_{i}+\epsilon_{i,m,t}
	%   \label{model_ind_2}
	%\end{equation}

	% Please add the following required packages to your document preamble:
% \usepackage{booktabs}
% \usepackage{graphicx}
\begin{table}[H]
\centering
\begin{tabular}{@{}lccccc@{}}
\toprule
Variables               & N     & Mean    & Standard Deviation & Min    & Max   \\ \midrule
Rating\_Entropy\_Self   & 4,562 & 0  & 0.217  & -0.0985 & 0.9015     \\
Rating\_Entropy\_Others & 4,562 & 0   & 0.183  & -0.227  & 0.773     \\
Average\_Rating\_Self   & 4,562 & 4.531   & 1.316  & 1      & 5     \\
Average\_Rating\_Others & 4,562 & 4.88    & 0.205  & 1      & 5     \\
Review\_Count           & 4,562 & 5.384   & 6.836  & 0      & 52    \\
Experience              & 4,562 & 1.758   & 0.929  & 0      & 3.714 \\
Price\_Difference       & 4,562 & -0.0333 & 0.392  & -2.171 & 3.139 \\
Market\_LogRevenue      & 4,562 & 12.24   & 7.9    & 0      & 22.3  \\ \bottomrule
\end{tabular}%
\caption{Summary Statistics - Installer Level Analysis}
\label{sumstats_ind}
{\footnotesize \textit{Note: All entropy variables are demeaned.}}
\end{table} 
	
	% Please add the following required packages to your document preamble:
% \usepackage{booktabs}
% \usepackage{graphicx}
\begin{table}[H]
\centering
\begin{tabular}{@{}lllllllll@{}}
\toprule
Variables                   & (1)    & (2)    & (3)    & (4)    & (5)    & (6)    & (7)        & (8)     \\ \midrule
(1) Rating\_Entropy\_Self   & 1      & \multicolumn{7}{l}{}                                              \\
(2) Rating\_Entropy\_Others & -0.094 & 1      & \multicolumn{6}{l}{}                                     \\
(3) Average\_Rating\_Self   & -0.088 & 0.069  & 1      & \multicolumn{5}{l}{}                            \\
(4) Average\_Rating\_Others & 0.025  & -0.523 & -0.017 & 1      & \multicolumn{4}{l}{}                   \\
(5) Review\_Count           & 0.241  & 0.061  & 0.205  & -0.039 & 1      & \multicolumn{3}{l}{}          \\
(6) Experience              & -0.001 & 0.143  & 0.036  & -0.062 & 0.127  & 1      & \multicolumn{2}{l}{} \\
(7) Price\_Difference       & -0.015 & -0.043 & 0.003  & 0.016  & -0.029 & -0.027 & 1          &         \\
(8) Market\_LogRevenue      & -0.026 & 0.006  & -0.029 & 0.044  & -0.086 & 0.451  & -0.059     & 1       \\ \bottomrule
\end{tabular}%
\caption{Correlation Matrix - Installer Level Analysis}
\label{corr_ind}
\end{table} 
	
	Tables \ref{sumstats_ind} and \ref{corr_ind} above present the summary statistics and the correlation matrix. By Table \ref{corr_ind}, correlations among explanatory variables are relatively low and do not hurt the validity of regression analysis. We also checked VIF scores of variables, and verified that they are all within the suggested range (\cite{hairmultivariate}), providing further support on the validity of our analysis.
	

	
	\subsection{Results}
	
Columns (I) through (III) of Table \ref{reg_ind_all} present our estimation results for three specifications. Column (III) includes the estimates under the regression model \eqref{model_ind_3}, while the estimates in other columns are obtained by considering only some of those explanatory variables in the regression. In particular, we obtain the estimates in column (I) by excluding all entropy variables from \eqref{model_ind_3}, and the estimates in column (II) by excluding variables related to the installer's rating entropy from \eqref{model_ind_3}. Table \ref{reg_ind_all} identifies three key results.

% Please add the following required packages to your document preamble:
% \usepackage{booktabs}
% \usepackage{graphicx}
\begin{table}[]
\centering
\begin{threeparttable}
\begin{tabular}{@{}lccc@{}}
\toprule
                  & (I)      & (II)    & (III)   \\  
& Installer's Activity & Installer's Activity & Installer's Activity \\
Variables & Level    & Level   & Level   \\ \midrule
Rating\_Entropy\_Self                                &                            &                            & 1.890***                   \\
                                                     &                            &                            & (0.000)                    \\
Rating\_Entropy\_Self$^2$                            &                            &                            & -3.473***                  \\
                                                     &                            &                            & (0.000)                    \\
Rating\_Entropy\_Others                              &                            & 0.524**                    & 0.488**                    \\
                                                     &                            & (0.004)                    & (0.007)                    \\
Rating\_Entropy\_Others$^2$                          &                            & -2.533***                  & -2.593***                  \\
                                                     &                            & (0.000)                    & (0.000)                    \\
Average\_Rating\_Self                                & -0.876***                  & -0.834***                  & -0.774**                   \\
                                                     & (0.000)                    & (0.000)                    & (0.001)                    \\
Average\_Rating\_Others                              & 0.000618                   & 0.000236                   & 0.000964                   \\
                                                     & (0.975)                    & (0.991)                    & (0.963)                    \\
Review\_Count                                        & 0.0561***                  & 0.0534***                  & 0.0472***                  \\
                                                     & (0.000)                    & (0.000)                    & (0.000)                    \\
Experience                                           & 0.218***                   & 0.212***                   & 0.206***                   \\
                                                     & (0.000)                    & (0.000)                    & (0.000)                    \\
Price\_Difference                                    & 0.0593                     & 0.0690                     & 0.0722                     \\
                                                     & (0.487)                    & (0.425)                    & (0.402)                    \\
Market\_LogRevenue                                   & -0.0168***                 & -0.0169***                 & -0.0159***                 \\
                                                     & (0.000)                    & (0.000)                    & (0.001)                    \\
Constant                                             & 2.690***                   & 2.803***                   & 2.961***                   \\
                                                     & (0.000)                    & (0.000)                    & (0.000)                    \\
Observations                                         & 4562                       & 4562                       & 4562                       \\
Fixed Effect                             & Yes        & Yes        & Yes       \\
State Dummies                            & Yes        & Yes        & Yes       \\
Adjusted R$^2$                                                     & 0.627                      & 0.630                      & 0.633       \\
AIC                                                  & 13267.2                    & 13234.9                    & 13205.7                    \\
BIC                                                  & 13325.0                    & 13305.6                    & 13289.2                    \\ \bottomrule

\end{tabular}%
\begin{tablenotes}
\item Note: $p$-value in parentheses; $^\star p<0.05;^{\star\star} p<0.01;^{\star\star\star} p<0.001 $
\end{tablenotes}
\caption{Installer Level Analysis}
\label{reg_ind_all}
\end{threeparttable}
\end{table}


First, the set of variables representing ``noise'' or dispersion of ratings have a significant impact on an installer's activity level in the marketplace. This is because  all entropy variables are found to be statistically significant in the column (III) of Table \ref{reg_ind_all}.
	
	Second, an installer's rating dispersion has a positive and statistically significant first-order effect on the installer's activity level because in the column (III), the variable ``Rating\_Entropy\_Self'' is found to be significant and its coefficient is positive ($\beta_{1} = 1.890,p<0.001$). On the other hand, we also find that an installer's rating dispersion has a negative and statistically significant second-order effect on its activity level. This is because in the column (III), the variable ``Rating\_Entropy\_Self$^2$'' is significant and its coefficient is negative ($\beta_{2} = -3.473, p<0.001$). Combining these two effects, an installer's rating dispersion has a concave and non-monotone impact on its activity level in the online marketplace. Specifically,  an installer's rating dispersion increases its activity level if and only if the aforementioned dispersion is below a certain threshold; otherwise, any additional dispersion in the installer's ratings lowers its activity level in the marketplace. This finding offers support for Hypothesis 1A if and only if the installer's rating entropy is smaller than the threshold; otherwise, our findings are in support of Hypothesis 1B.
	
	Third,  our estimation shows that the entropy of competitors' ratings impacts an installer's activity level in the same way as the entropy of the installer's ratings. Specifically, it has a positive and significant first-order effect (as ``Rating\_Entropy\_Others'' is significant and its coefficient $\beta_{3} =  0.488$ ($p<0.01$)), and a negative and significant second-order effect (as ``Rating\_Entropy\_Others$^{2}$'' is significant and its coefficient $\beta_{4} = -2.593, p<0.001$). Combining these two effects,
	the dispersion in competitors' ratings increases the installer's activity level if and only if the aforementioned dispersion is below a threshold. When the dispersion of competitors' ratings is above that threshold, any additional dispersion in competitors' ratings lowers the installer's marketplace activity. This implies support for Hypothesis 2A if and only if the competitors' rating entropy is below the threshold; otherwise, our findings offer support for Hypothesis 2B.
	
	
	Figures \ref{fig: marginsplot_ind_ent_self} and  \ref{fig: marginsplot_ind_ent_others} illustrate the explained nonlinear effects of the rating entropy on the installer's activity level in the online marketplace. In generating Figures \ref{fig: marginsplot_ind_ent_self} and  \ref{fig: marginsplot_ind_ent_others}, we use the estimated regression coefficients in the column (III) of Table \ref{reg_ind_all}. As is apparent from these figures, on average, the installer's activity level first increases and then decreases with its rating entropy (or the rating entropy of its competitors), yielding an inverted U-shaped relationship between the two.
	
	\begin{figure}
		\centering
		\includegraphics[width=0.7\linewidth]{marginsplot_entself.png}
		\caption{Margins plot for installer's rating entropy versus its activity level with 95\% confidence interval}
		\label{fig: marginsplot_ind_ent_self}
	\end{figure}
	
	\begin{figure}
		\centering
		\includegraphics[width=0.7\linewidth]{marginsplot_entothers.png}
		\caption{Margins plot for other installer's rating entropy versus the installer's activity level with 95\% confidence interval}
		\label{fig: marginsplot_ind_ent_others}
	\end{figure}
	
	
	
	
	
	Finally, in all three columns of Table \ref{reg_ind_all}, the installer's average rating is significantly and negatively linked with its activity level. Put another way, installers appear to extend fewer proposals as their average ratings increase. One reason for this behavior could be that  installers become more selective after they attain a high average rating in the marketplace. Selectiveness can emerge because the installers might think that with a higher average rating, their proposals are more likely to be accepted by customers, and thus making too many offers increases their likelihood of coming across with a negative customer.
	
	\section{Market-Level Analysis \& Results} \label{Sec: Market-level}
	
	An important performance metric for the marketplace operator is the number of matches (i.e., agreed proposals) between installers and customers in the marketplace. This section estimates how the market-level rating dispersion impacts \emph{market transaction} that is defined as the logged number of matches in the market. With this, we test Hypotheses 3A and 3B in Section \ref{Sec: Hypothesis}.
	
	We will only use numerical ratings in this section. Later, Section \ref{Sec: TextMining} will account for text reviews in the market-level analysis. We will provide various additional robustness checks of our findings, and address potential endogeneity concerns in an extended model in Section \ref{Sec: Robustness}.
	
	
	
	Recalling that markets and months are indexed by $m$ and $t$, respectively, we use the following regression equation for the estimation:
	\begin{align} \nonumber
	\text{Market\_Transaction}_{m,t+1} & =\beta_{5} + \beta_{6} \text{Rating\_Entropy\_Mkt}_{m,t}+ \beta_{7} \text{Rating\_Entropy\_Mkt}_{m,t}^2\\ \label{reg: market-level-rating}
	&+ \text{Controls}_{m,t}  + \xi_{m} + \epsilon_{m,t+1}.
	\end{align}
	Here, $\xi_{m}$ is a market-level fixed effect, and it represents the time-invariant market-specific factors that may influence the market transaction.

	Performing \eqref{reg: market-level-rating} requires us to convert the installer-level monthly panel data to the market-level monthly panel data based on the markets defined in Section \ref{defining_local_market}. Our data include the number of agreed proposals for each installer $i$ in each month $t$. To create our dependent variable $\text{Market\_Transaction}_{m,t+1}$, we first calculate the total number of proposals accepted by customers in market $m$ and month $t+1$, and then take the natural logarithmic transformation of that sum. Formally, $\text{Market\_Transaction}_{m,t+1} =$ $\ln\left( \sum_{i \in \text{Market\ } m} \text{Successful\_Proposals}_{i}+ 1 \right)$ in \eqref{reg: market-level-rating}.  We employ this standard transformation because the number of matches is right-skewed and the transformation increases the normality of errors. (As a robustness check, we also perform the analysis without log transformation and the results are consistent.)
	
	A key explanatory variable in \eqref{reg: market-level-rating} is $\text{Rating\_Entropy\_Mkt}_{m,t}$, which is the entropy of all installers' ratings up to (and including) month $t$ in the market $m$.
	Because we have two competing hypotheses about the impact of market-level rating entropy (i.e., Hypotheses 3A and 3B in Section \ref{Sec: Hypothesis}), we allow for a  nonlinear relationship between the market-level rating entropy and the dependent variable. Thus, \eqref{reg: market-level-rating} also contains the quadratic term  $\text{Rating\_Entropy\_Market}_{m,t}^{2}$. The aforementioned two variables are our main explanatory variables, and $\beta_{6}$ and $\beta_{7}$ are the key coefficients of interests. The values of these coefficients together with the significance of the associated explanatory variables will help us determine how the market-level rating entropy impacts market transactions.
	
	
	In \eqref{reg: market-level-rating}, $\epsilon$ is the market-level error term, and represents random factors that are unobservable in the data and affect market transactions. We also use various control variables ($\text{Controls}_{m,t}$) in \eqref{reg: market-level-rating}. We control for the state of the market. To do that, we created 33 state dummies to represent 33 different states included in the data set. In our dataset, 18\% of markets span across more than one state.  In light of this, each state dummy represents the fraction of installers that are located in that state within the market $m$. For example, suppose market 1 has 25\% of installers from state X and 75\% from Y. Then, we assign 0.25 to the dummy variable State\_X and 0.75 to the dummy variable State\_Y for market 1.  Similar to the installer-level analysis in Section \ref{Sec: Installer-level}, we created the variable $\text{Average\_Experience}_{m,t}$ that represents the average experience of installers in the market $m$ up to and including month $t$. In parallel to the installer-level analysis, we use the variable $\text{Average\_Rating\_Mkt}_{m,t}$ to control for the average rating of all installers in the market $m$ until (and including) month $t$. We also control for the difference between the average unit price of installed 1 KW solar system in the marketplace and off-marketplace, which is represented by the variable $\text{Price\_Difference\_Mkt}_{m,t}$. Finally, we use $\text{Market\_LogRevenue}_{m,t}$ as a control where it is as defined in Section \ref{Sec: Installer-level}. Summary statistics can be found in Table \ref{sumstats_mkt}; the correlation coefficients among variables are presented in Table \ref{corr_mkt}.  We also checked VIF scores of variables, and verified that they are all within the suggested range (\cite{hairmultivariate}).

% Please add the following required packages to your document preamble:
% \usepackage{booktabs}
% \usepackage{graphicx}
\begin{table}[H]
\centering

\begin{tabular}{@{}lccccc@{}}
\toprule
Variables            & N   & Mean   & Standard Deviation & Min    & Max   \\ \midrule
Rating\_Entropy\_Mkt & 642 & 0  & 0.236              & -0.191  & 0.809    \\
Market\_LogRevenue   & 642 & 7.887  & 8.102               & 0      & 22.3  \\
Average\_Rating\_Mkt & 642 & 4.870  & 0.245              & 3      & 5     \\
Average\_Experience  & 642 & 1.426 & 1.110              & 0      & 3.332    \\
Price\_Difference\_Mkt    & 642 & -0.0106 & 0.146              & -0.504 & 1.312 \\ \bottomrule
\end{tabular}%

\caption{Summary Statistics - Market Level Analysis}
\label{sumstats_mkt}
\end{table} 
	% Please add the following required packages to your document preamble:
% \usepackage{booktabs}
% \usepackage{graphicx}
\begin{table}[H]
\centering
\begin{tabular}{@{}llllll@{}}
\toprule
Variables                & (1)    & (2)    & (3)    & (4)    & (5) \\ \midrule
(1) Rating\_Entropy\_Mkt & 1      &        &        &        &     \\
(2) Average\_Rating\_Mkt & -0.624 & 1      &        &        &     \\
(3) Average\_Experience  & 0.198  & -0.107 & 1      &        &     \\
(4) Price\_Difference\_Mkt    & -0.018 & -0.041 & -0.012 & 1      &     \\
(5) Market\_LogRevenue   & 0.157  & -0.041 & 0.498  & -0.126 & 1   \\ \bottomrule
\end{tabular}
\caption{Correlation Matrix - Market Level Analysis}
\label{corr_mkt}
\end{table} 
	
% We calculated this by averaging installers' experience levels across the market $m$. As another control, we created the variable $\text{Review\_Count\_Mkt}_{m,t}$ that measures the total number of reviews by all installers in the market $m$ up to and including month $t$.	
	
	\subsection{Results}
	
	
	Table 6 presents our regression estimates for two specifications. Column (I) shows the estimates obtained with the regression \eqref{reg: market-level-rating} in the absence of $\text{Rating\_Entropy\_Mkt}_{m,t}$ and $\text{Rating\_Entropy\_Mkt}^2_{m,t}$ variables, while column (II) includes the estimates obtained by running the regression \eqref{reg: market-level-rating} considering all variables in \eqref{reg: market-level-rating}.
	
	% Please add the following required packages to your document preamble:
% \usepackage{booktabs}
% \usepackage{graphicx}
\begin{table}[]
\centering
\begin{threeparttable}[t]
\begin{tabular}{@{}lcc@{}}
\toprule
                                    & (I)                   & (II)                  \\
Variables                                    & Market Transaction & Market Transaction\\ \midrule
Rating\_Entropy\_Mkt                         &                       & 1.060***              \\
                                             &                       & (0.000)               \\
Rating\_Entropy\_Mkt$^2$                     &                       & -1.610***             \\
                                             &                       & (0.000)               \\
Average\_Rating\_Mkt                         & -0.273                & -0.185                \\
                                             & (0.071)               & (0.212)               \\
Average\_Experience                          & 0.0197*               & 0.0138                \\
                                             & (0.013)               & (0.073)               \\
Price\_Difference\_Mkt                       & 0.239                 & 0.287                 \\
                                             & (0.282)               & (0.194)               \\
Market\_LogRevenue                           & -0.0846               & -0.0663               \\
                                             & (0.057)               & (0.123)               \\
Constant                                     & 3.443**               & 2.909*                \\
                                             & (0.009)               & (0.025)               \\
Market Fixed Effect                          & Yes                   & Yes                   \\
Weighted State Dummies                       & Yes                   & Yes                   \\
Observations                                 & 642                   & 642                   \\
Adjusted R$^2$                                  & 0.739                 & 0.747                 \\
AIC                                          & 1075.6                & 1059.6                \\
BIC                                          & 1156.0                & 1148.9                \\ \bottomrule
\end{tabular}%
\begin{tablenotes}
\item Note: $p$-value in parentheses; $^\star p<0.05;^{\star\star} p<0.01;^{\star\star\star} p<0.001 $
\end{tablenotes}
\caption{Market Level Analysis}
\end{threeparttable}
\label{reg_mkt_simplified}
\end{table} 
	
	
	
	Regression estimates reveal the following key findings. First, market-level rating dispersion has a significant and positive first-order effect on the market transaction. This is because the coefficient of ``$\text{Rating\_Entropy\_Mkt}$'' is positive ($\beta_{6}=1.060$) and statistically significant ($p<0.001$) in the column (II) of Table 6. The market-level rating dispersion also has a significant and negative second-order effect on the market transaction as the coefficient of the quadratic term ``$\text{Rating\_Entropy\_Mkt}^2$'' is negative ($\beta_{7}=-1.160$) and statistically significant ($p<0.001$) in the column (II). Combining these two effects, regression estimates indicate a concave and non-monotone relationship between the market-level rating dispersion and the market transaction.  Specifically, our findings indicate that the market-level rating dispersion increases the market transaction if and only if the mentioned dispersion is smaller than a threshold, for any dispersion beyond that threshold, an increase in the market-level rating dispersion dampens the market transaction (and number of matches). These findings support Hypothesis 3A if and only if the market-level rating dispersion is below a certain threshold; otherwise, our results are in support for Hypothesis 3B.
	
	Figure \ref{fig: marginsplot_mkt_entmkt} further illustrates this nonlinear relationship via a margins plot using coefficients generated from estimates in the column (II) of Table 6. As we observe from this figure, on average, the market transaction first increases then decreases with the market-level rating dispersion. Same effect is also valid for the number of matches.
	
	Finally, market-level estimates in Table 6 suggest that after controlling for market conditions, installer experience, price, and state, the average rating is not significantly associated with the market transaction.
	\begin{figure}
		\centering
		\includegraphics[width=0.7\linewidth]{marginsplot_entmkt.png}
		\caption{Margins plot for the market-level rating entropy versus the market transaction with 95\% confidence interval}
		\label{fig: marginsplot_mkt_entmkt}
	\end{figure}
	
	
\section{Text Mining}\label{Sec: TextMining}

	
	In this section, we incorporate various methods to leverage rich text information in reviews.  First, we will use the state-of-the-art advanced natural language processing (NLP) technique called BERT (i.e., \textit{Bidirectional Encoder Representations from Transformers}) to measure the dispersion in text reviews. The BERT method was developed by Google AI in 2018, and incorporated by  Google Search Engine in late 2019 \citep{devlin2018bert,BERT}. To the best of our knowledge, there is no other paper in operations management literature that considers this deep learning based technique. Second, we will use another state-of-the-art NLP method to generate the sentiment score of each review. As the final step, we will incorporate these text-based metrics in our regression analysis to derive insights related to the text content.
	
\subsection{Measuring Text-Based Review Dispersion via Natural Language Processing Technique BERT} \label{Subsec: Define Txt Ent}
	
	In our data set, we have $3607$ pieces of text reviews, and we apply the following five steps to measure the dispersion in text reviews. First, we use BERT to convert each text review to a semantics-sensitive numerical vector. We will provide more details about this method later in this section.  As a second step, we normalize each vector to unit length. Third, we measure the cosine similarity between any two review vectors to find the context similarity between the two. The cosine similarity between two normalized vectors $V_{1}$ and $V_{2}$ equals the inner product of two, i.e., $V_1 \cdot V_2$, and gives the cosine of the angle between the two vectors. The angle represents the similarity in the orientation of two review vectors. For example, if the angle is $0$, the vectors are at the same orientation and hence the similarity is the maximum. The second and third steps above are standard in identifying the similarity between two vectors (see, e.g., \cite{hoberg2016text}). As the fourth step, we identify the cosine distance between every two review vectors using the fact that the cosine distance between the two normalized vectors $V_{1}$ and $V_{2}$ equals $1$ minus the cosine similarity between the two. This distance reflects how much two reviews differ from each other. As the fifth and final step, we calculate the dispersion of text reviews, i.e., \emph{text-based dispersion}, for a review set of interest by enumerating all pairwise cosine distances of reviews in that set and taking their statistical median (the 50$^{\text{th}}$ percentile). For example, for a set of 10 text reviews, we have 45 (=$\binom{10}{2}$) pairwise cosine distances. Finding the text-based dispersion for this set requires computing the median of these 45 distances. If these 10 pieces of texts are dissimilar from each other, they contain richer information and the median of these $45$ distances shall be higher; and vice versa. (Section \ref{Sec: Mean} shows that our findings remain to be valid when the mean (rather than median) of cosine distances is considered in the fifth step of text-based dispersion measurement.)
	
	As a result of this procedure, similar to the rating entropy, we create the following three variables, each measuring the text-based dispersion in a different dimension: (i) $\text{Text\_Dispersion\_Self}_{i,t}$: Demeaned dispersion in installer $i$'s own text reviews up to and including month $t$. %Given $N_{i,t}$ pieces of text reviews available up to month $t$, it is calculated by computing $N_{i,t}\times (N_{i,t}-1)/2$ cosine distance pairs and taking the 50$^{th}$ percentile.
	(ii) $\text{Text\_Dispersion\_Others}_{i,t}$: Demeaned dispersion in text reviews of all other installers in the installer $i$'s market up to and including month $t$. %Given $N_{-i,t}$ pieces of text reviews of other installers up to month $t$, it is calculated by computing $N_{-i,t}\times (N_{-i,t}-1)/2$ cosine distance pairs and taking the 50$^{th}$ percentile.
	(iii) $\text{Text\_Dispersion\_Mkt}_{m,t}$: Demeaned dispersion in text reviews of all installers in the market $m$ up to and including month $t$. %Given $N_{m,t}$ reviews available in market $m$ up to month $t$, we calculate $N_{m,t}\times (N_{m,t}-1)/2$ cosine distance pairs and take the 50$^{th}$ percentile to calculate the variable.
	Note that each of these variables is centered around its mean. We apply this standard procedure because in addition to these terms, we will also consider quadratic terms in our regression.
	
	We now elaborate the BERT model we used to \textit{vectorize} the text reviews. BERT is a natural language processing (NLP) model that transforms texts into numeric vectors while also preserving the meaning of texts. It belongs to the category of NLP methods that performs word embedding. In literature, in different contexts than ours, text data are commonly vectorized based on word counts, ignoring the semantics and word ordering (see, e.g., \cite{hoberg2016text} and \cite{loughran2011liability}). However, our context involves texts that are informal writings and often contain emotions. Simply capturing word frequencies does not provide accurate results if similar emotions can be expressed with synonymous words.  Thus, our analysis requires a vectorization that preserves the information and sentiment of the text reviews despite the use of synonyms and/or different styles. The BERT model achieves that. Specifically, the BERT model has two distinct advantages. First, it understands the semantics. For example, consider the 3 sentences:
	\begin{align*}
	{\footnotesize
		\text{Sentence 1: \texttt{they did a good job.}} \quad \text{Sentence 2: \texttt{they did an awful job.}} \quad \text{Sentence 3: \texttt{they did a great job.}}
	}
	\end{align*}
	Considering the meaning of the sentences, we expect the distance between sentences 1 and 3, $D(1,3)$, to be smaller than the distance between 2 and 3 or 1 and 2, i.e., $D(2,3)$ or $D(1,2)$. The BERT model vectorization enables just that; it projects ``good'' and ``great'' to vectors that are closer to each other. In this example, with BERT, we have $D(1,3) = 0.03 < D(1,2) = 0.09 <D(2,3) = 0.1$. This level of distinction is not feasible without word embedding (e.g., by simply using a word counter vectorizer).
	
	Second, the BERT model takes word ordering into account. For example, the two sentences ``The food was good, not bad at all'' and ``The food was bad, not good at all'' have the opposite meanings. Common vectorization methods (e.g., ``bag-of-words'' approach) are not able to capture this difference as words and number of counts are the same in both sentences. But, the BERT model can easily differentiate between these two sentences.
	

	
	\subsection{Sentiment Scores of Text Reviews} \label{Subsec: Sentiment}
	
	We use the VADER model to generate sentiment scores for text reviews. VADER is short for \emph{Valence Aware Dictionary and sEntiment Reasoner} and developed by  \cite{hutto2014vader} as a ``parsimonious rule-based model for sentiment analysis of social media text.'' Since review text shares many structural similarities with the social media text, an important application area of this model is the text analysis of reviews. For each text review, VADER produces a sentiment intensity score from -1 to 1, with 1 representing very positive and -1 representing very negative sentiments.
	
	VADER has key advantages. In contrast to models that use a polarized lexicon where a word is classified as either positive, negative or neutral, VADER is sensitive to both polarity and strength of the sentiment.  The method also understands conventional syntactical and grammatical components in the text and reflects them in the sentiment intensity score it generates. Among other features, the method accounts for the exclamation mark, capitalization especially the usage of all-caps, degree adverbs such as ``extremely'' and ``marginally'', the contrastive conjunction (e.g.,``but''), conventional emojis, slangs and emoticons in its sentiment intensity score calculation.
	
	
	For example, the following review, which was rated as 5-star, received a sentiment score of 0.8622 under the VADER method:
	
	``\textit{Mike at ($\ldots$) was friendly, courteous, professional and very helpful.  At first I did not know what kind of system I wanted, because my roof was too small and I had some trees in the way.  Mike had never installed a tracking system but he did recommend it.  It seemed like we would get the best ``bang for the buck'' with this system, so I went with it.  Mike had all subcontractor there on time as well as all the equipment.  It was up and running in less than a week.  I love it.}''
	
	As another example, the following review, which was rated as 1-star, received a sentiment score of -0.7184 under the VADER method:
	
	``\textit{Do not hire ($\ldots$)  to install a solar system. Do not hire ($\ldots$) to do anything. Evan and all his various companies and names  ARE NOT LICENCED OR INSURED. I was scammed by Mr. Evan ($\ldots$) in December of 2013. He installed the system wrong and incomplete even though all the parts and materials were provided for him. Please take the time to do your research and check references and validate licenses and insurance information. It will save u more money than to trust a cheap con artist. All the info at ($\ldots$)  is fraudulent lies. Evan ($\ldots$) is also known as ($\ldots$).}''
	
	Based on this method, we created three variables that represent average sentiment intensity scores in three dimensions: (i) $\text{Average\_Sentiment\_Self}_{i,t}$: The average sentiment intensity score of installer $i$'s all text reviews up to and including month $t$. (ii) $\text{Average\_Sentiment\_Others}_{i,t}$: The average sentiment intensity score of competitors' text reviews up to and including month $t$ in the installer $i$'s market. (iii) $\text{Average\_Sentiment\_Mkt}_{m,t}$: The average sentiment intensity score of all text reviews in the market $m$ up to and including month $t$.

%% Please add the following required packages to your document preamble:
% \usepackage{booktabs}
% \usepackage{graphicx}
\begin{table}[H]
\centering
\begin{tabular}{@{}lccccc@{}}
\toprule
Variables                   & N     & Mean      & Standard Deviation & Min     & Max   \\ \midrule
Text\_Dispersion\_Self   & 4,562 & 0.0337    & 0.0711             & -0.0636 & 0.318 \\
Text\_Dispersion\_Others & 4,562 & -0.000235 & 0.0219             & -0.0609 & 0.241 \\
Average\_Sentiment\_Self    & 4,562 & 0.221     & 0.239              & -1.239  & 0.546 \\
Average\_Sentiment\_Others  & 4,562 & 0.0669    & 0.205              & -1.161  & 0.396 \\\bottomrule
\end{tabular}%
\caption{Summary Statistics - Individual Level Text-based Variables}
\label{sumstats_ind_textbased}
\end{table} 


\begin{table}[H]
\centering
\begin{tabular}{@{}lccccc@{}}
\toprule
Variables                   & N     & Mean      & Standard Deviation & Min     & Max   \\ \midrule
Text\_Dispersion\_Self   & 4,562 & 0      & 0.0711             & -0.064 & 0.318 \\
Text\_Dispersion\_Others & 4,562 & 0    & 0.0219             & -0.061 & 0.241 \\
Average\_Sentiment\_Self    & 4,562 & 0.411     & 0.358             & -0.827  & 0.958 \\
Average\_Sentiment\_Others  & 4,562 &0.561    & 0.283              & -0.599  & 0.958 \\
Text\_Dispersion\_Mkt	    & 642	&0	    &0.036	             &-0.062   &0.235\\
Average\_Sentiment\_Mkt	    & 642	& 0.463  	& 0.334	            & 0.0593   & 0.866 \\ \bottomrule
\end{tabular}%
\caption{Summary statistics of text-based variables for installer and market-level analysis}
\label{sumstats_textbased}
\end{table} 
	
\subsection{Empirical Analysis Using Variables Derived From Text Mining}
	
	We now discuss the analysis we conducted with the text-based dispersion and average sentiment scores we constructed in Sections \ref{Subsec: Define Txt Ent} and \ref{Subsec: Sentiment}. With these additional variables, we aim to examine the following questions: (i) Is the text content significant in explaining installers' activity levels and market transactions? (ii) How do the text-based dispersion and the average sentiment intensity score influence an installer's activity level and market transactions? %By examining these questions, we will also test Hypotheses 1A, 1B, 2A, 2B, 3A and 3B in the context of text reviews.
To study these questions, we consider the following regression models:
	\begin{align}  \nonumber
	& \text{Installer\_Activity}_{i,m,t+1} \\ \nonumber
	& = \theta_{0}+ \theta_{1} \text{Rating\_Entropy\_Self}_{i,t}+ \theta_{2} \text{Rating\_Entropy\_Self}_{i,t}^ {2} + \theta_{3} \text{Rating\_Entropy\_Others}_{i,t} \\ \nonumber
	& + \theta_{4} \text{Rating\_Entropy\_Others}_{i,t}^{2} + \theta_{5} \text{Text\_Dispersion\_Self}_{i,t}+  \theta_{6}  \text{Text\_Dispersion\_Self}_{i,t}^ {2}  \\ \label{model_ind_textbased}
	&+ \theta_{7}  \text{Text\_Dispersion\_Others}_{i,t} + \theta_{8} \text{Text\_Dispersion\_Others}_{i,t}^{2}  + \text{Controls}_{i,m,t}+ \alpha_{i} + \epsilon_{i,t+1}.\\ \nonumber
	%\end{align}
	%\begin{align} \nonumber
	& \text{Market\_Transaction}_{m,t+1} \\ \nonumber
	& =  \eta_{0} + \eta_{1} \text{Rating\_Entropy\_Mkt}_{m,t}+  \eta_{2} \text{Rating\_Entropy\_Mkt}_{m,t}^2 \\ \label{reg: market-level-textbased}
	&+ \eta_{3} \text{Text\_Dispersion\_Mkt}_{m,t}+ \eta_{4} \text{Text\_Dispersion\_Mkt}_{m,t}^2  + \text{Controls}_{m,t}  + \xi_{m} + \epsilon_{m,t+1}.
	\end{align}
	
	In \eqref{model_ind_textbased} and \eqref{reg: market-level-textbased}, all variables except text-based variables and control variables on average rating are the same as the ones in \eqref{model_ind_3} and \eqref{reg: market-level-rating}, respectively. An installer's text-based dispersion correlates with its rating entropy on a lower level (correlation coefficient is 0.304; see Tables \ref{corr_measures_dispersion_text_and_rating} and \ref{corr_ind_text} for the entire correlation matrix). Thus, \eqref{model_ind_textbased} and \eqref{reg: market-level-textbased} consider the text-based dispersion and rating entropy variables in the same regression. On the other hand, by Table \ref{corr_measures_dispersion_text_and_rating}, an installer's average sentiment intensity score significantly correlates with its average rating (with a correlation coefficient of 0.832). As a result, in \eqref{model_ind_textbased} and \eqref{reg: market-level-textbased},  we will use average sentiment intensity scores and average rating variables as substitute controls. Table \ref{reg_ind_withtext} shows the estimation results obtained by using different set of explanatory variables in \eqref{model_ind_textbased}. In each column of Table \ref{reg_ind_withtext}, blank represents the absence of the corresponding explanatory variable in the regression.
	
	%\begin{figure}
	%	\centering
	%	\includegraphics[width=0.7\linewidth]{sentscore_violin.png}
	%	\caption{Sentiment Scores and Ratings}
	%	\label{fig: sentscore_violinr}
	%\end{figure}
	% Figure \ref{fig: regplot_text_d_ent_others} displays the relationship between an installer's text-based dispersion and its rating entropy via a scatter plot on a regression line.
	%\begin{figure}
	%	\centering
	%	\includegraphics[width=0.7\linewidth]{regplot_text_d_ent_others.png}
	%	\caption{Text-based Dispersion and Entropy of Ratings}
	%	\label{fig: regplot_text_d_ent_others}
	%\end{figure}
	

	
	Regression estimates in Table \ref{reg_ind_withtext} reveal key findings. First, an installer's or its competitors' rating entropy continues to have an inverted U-shaped impact on the installer's activity level even when corresponding text-based dispersion variables are also considered.  Second, an installer's text-based dispersion has a significant and positive first-order effect on the installer's activity level, whereas it has a significant and negative second-order effect on the installer's activity level (as $\theta_{5} > 0$ and $\theta_{6} < 0$ are both found to be significant). Combining two, an installer's text-based dispersion also has an inverted U-shaped impact on its activity level. Third, as indicated by insignificance of coefficients for ``\text{Average\_Sentiment\_Score\_Self}'' and ``\text{Average\_Sentiment\_Score\_Others}'', installer's average sentiment score or its competitors' average sentiment score does not have a significant impact on the installer's activity level.
	
	%Fourth, when the text content and numerical ratings are considered together, competitors' rating entropy has an insignificant impact on the installer's activity level.
	
	
	Table \ref{reg_mkt_textbased} displays the estimates obtained by using different set of explanatory variables in \eqref{reg: market-level-textbased}. There are three key findings. First, the  market-level rating entropy continues to be significant and have an inverted U-shaped impact on market transactions (and number of matches). Thus, our original finding in Section \ref{Sec: Market-level} is robust.  Second, although the text-based dispersion has an inverted U-shaped relationship with market transactions, it has an insignificant impact on the market transaction.  Third, the average market-level sentiment score has a significant and negative impact on the market transaction.
	
	% installers' activity levels and market transactions can also be attributed to sentiment scores and text-based dispersion measures. We use variables $\text{Avg\_Sent\_Self}_{i,t}$ in place of $\text{Avg\_Rating\_Self}_{i,t}$, $\text{Avg\_Sent\_Others}_{i,m,t}$ in place of $\text{Avg\_Others}_{i,m,t}$ to represent individual ratings in individual level analysis, $\text{Avg\_Sent\_Mkt}_{m,t}$ in place of $\text{Avg\_Mkt}_{m,t}$ in market level analysis, and ran the same regression models. The results are presented in table \ref{reg_ind_withtext} for installer level analysis and table \ref{reg_mkt_textbased} for market level analysis.  \\
	
	
\section{Robustness Checks} \label{Sec: Robustness}
	
\subsection{Dynamic Panel Model}  \label{Sec: Dynamic Panel}
	
	The regression model in Section \ref{Sec: Installer-level} considers fixed effect for each installer, and that accounts for time-invariant installer-specific factors that may impact the dependent variable, i.e., installer's activity level. This section extends our regression model  \eqref{model_ind_3} to a dynamic panel model by including lagged dependent variables. The inclusion of these variables aims to consider any other unobserved heterogeneity that may influence the dependent variable but not captured in  \eqref{model_ind_3}. In light of this, for the installer-level analysis, our regression equation is extended to the following:
	\begin{align} \nonumber
	&\text{Installer\_Activity}_{i,m,t+1} \\ \nonumber
	&=\gamma_{0}+\gamma_{1} \text{Installer\_Activity}_{i,t}+ \gamma_{2}\text{Installer\_Activity}_{i,t-1}+
	\gamma_{3} \text{Rating\_Entropy\_Self}_{i,t} \\ \nonumber
	&+ \gamma_{4} \text{Rating\_Entropy\_Self}_{i,t}^ {2} + \gamma_{5} \text{Rating\_Entropy\_Others}_{i,t}  + \gamma_{6} \text{Rating\_Entropy\_Others}_{i,t}^{2} \\ \label{eq: extended_ind}
	&+ \alpha_{i}+ \text{Controls}_{i,m,t}+ \epsilon_{i,t+1}.
	\end{align}
However,  the inclusion of the lagged dependent variables in the presence of fixed effects may cause endogeneity bias, as such an addition may lead a correlation between the regressors and the error \citep{nickell1981biases}. To overcome this, we use \cite{arellano1991some}'s method that addresses endogeneity bias in dynamic panel data. \cite{arellano1991some} estimator is a general method of moments estimator that is based on dynamic panel data with first differences. It uses lagged variables as instruments to address the endogeneity bias. \cite{arellano1991some} estimation requires serially uncorrelated first-differenced errors. We provide support for this property in Table \ref{autocorrelation_test}. We also modify the market-level model \eqref{reg: market-level-rating} to include lagged dependent variables:
	\begin{align} \nonumber
	&\text{Market\_Transaction}_{m,t+1}\\ \nonumber
	& =\eta_{0}+ \eta_{1} \text{Market\_Transaction}_{m,t}+ \eta_{2} \text{Market\_Transaction}_{m,t-1} + \eta_{3} \text{Market\_Transaction}_{m,t-2} \\ \label{eq: ext_market_level}
	&+ \eta_{4} \text{Rating\_Entropy\_Mkt}_{m,t}+ \eta_{5}\text{Rating\_Entropy\_Mkt}_{m,t} ^2 + \xi_{m}+ \text{Controls}_{m,t}  +\epsilon_{m,t+1}.
	\end{align}
	Here, $\xi_{m}$ represents the time-invariant market-specific factors that may influence market transaction.  Table \ref{autocorrelation_test} also provides support for serially-uncorrelated first-differenced errors on the market-level. Thus, to overcome any potential endogeneity bias in this regression, we apply similar steps as the ones explained for the analysis of the installer-level activity. In applying  \cite{arellano1991some} estimator, we included $1-3$ lags of variables.
	
	Tables \ref{rob_ind_dynamic} and \ref{rob_mkt_dynamic} include our installer-level and market-level estimates. An installer's rating entropy continues to have an inverted U-shaped impact on the installer's activity level. Furthermore, there is also an inverted U-shaped relationship between the market-level rating entropy and market transaction (or number of matches).  Thus, our key findings are robust in this extension.
	
	
	
	%----
	
	%Staiger D, Stock JH (1997) Instrumental variables regression with weak instruments. Econometrica 65(3):557–586.
	%Nickell S (1981) Biases in dynamic models with fixed effects. Econometrica (6):1417–1426.
	%Arellano M, Bond S (1991) Some tests of specification for panel data: Monte Carlo evidence and an application to employment equations. The Review of Economic Studies 58(2):277–297
	%Bowsher C (2002) On testing overidentifying restrictions in dynamic panel data models. Economics Letters 77(2):211–220
	%
	%The regression in Eq. 1 includes county fixed effects, ci , to control for unobserved heterogeneity across counties that does not vary across time yet influences the staffing level. The presence of fixed effects in Eq. 1 along with the lagged dependent variables and regressors raises a concern of endogeneity bias in its estimation (Nickell (1981)). We use Arellano and Bond (1991) dynamic panel data model with first differences and lagged variables as instruments to overcome this issue. We limit the number of lags used as instruments in the model (Bowsher (2002)). To be specific, for the voting resource regressors (logVotersPerPW, logAbsentBallotsPerPP, logEarlyBallotsPerPP, logEDBallotsPerPP, logProvBallotsPerPP) we use the second and third lags as instruments (corresponding to both a midterm and presidential election). For voter demographic variables (PctDemocrat, PctWhite), we use the second election lag as an instrument, and for poll worker recruitment difficulty (PollDiff ), we use the most recent election lag as an instrument. We do not include lags of the other regressors (logPersonPerSqMile, HousePrice, MedInc, UseDRE, Pct65Plus) because we believe they should be uncorrelated with shocks to the number of voters or poll workers. The Hansen test (robust to heteroscedasticity) for overidentifying restrictions assumes a null hypothesis that our instruments meet the exogeneity requirement. We do not find evidence that the exogeneity assumption is violated (Table EC.2). We also address two issues with Arellano-Bond estimation. First, it may perform poorly if instruments are weak, which could occur if changes in county election demographics were fully adjustable from one election to the next, thereby having no relation to past values. We believe, however, that county demographics are somewhat rigid over time. Consistent with that view, we do not find evidence of weakness using F-statistics from the first-stage 2SLS regressions of the first differences of each endogenous variable (pooled across counties and election years) on its lagged instrument(s) (see Table EC.3). Second, Arellano-Bond estimation requires serially uncorrelated errors, which is supported (Table EC.2)
	%
	%To assess the validity of the instruments, we perform
	%several statistical tests to examine whether they
	%meet the relevance criteria. First, we note that the
	%R2 values from the first stage regressions of the four
	%endogenous variables lie between 0.55 and 0.70, indicating
	%that the instruments have significant explanatory
	%power. Second, the F -statistics of the excluded
	%instruments in the first stage regressions are well over
	%10 in all of our regressions, indicating that the instruments
	%are not “weak” in the sense of Staiger and Stock
	%(1997). Although not conclusive, these test statistics
	%
	%
	%\url{https://fmwww.bc.edu/GStat/docs/StataIV.pdf}
	%\url{https://www.univ-orleans.fr/deg/masters/ESA/CH/Geneve_Chapitre2.pdf}
	%\url{https://faculty.fuqua.duke.edu/econometrics/presentations/2013/Rossi-Instruments_and_Fixed_Effects.pdf}
	%
	%-----
	
	
	
	
	\subsection{Additional Support for Inverted U-Shaped Relationship}
	
	To further validate the inverted U-shaped relationship between an explanatory variable and the response variable, one must check whether the stationary point of the explanatory variable lies within its range in our sample. This check is important to distinguish the inverted U-shaped relationship from a concave monotone relationship.  The ranges of ``\text{Rating\_Entropy\_Self},'' ``\text{Rating\_Entropy\_Others}'' and ``\text{Rating\_Entropy\_Mkt}'' are provided in Tables \ref{sumstats_ind} and \ref{sumstats_mkt}.
	Based on our estimates in Sections \ref{Sec: Installer-level} and \ref{Sec: Market-level}, we calculate the stationary points for these variables as $S_{\text{self}} \doteq - \beta_{1}/ (2 \beta_{2}) = 0.272$, $S_{\text{others}} \doteq  - \beta_{3}/(2 \beta_{4}) =0.094 $ and $S_{\text{mkt}} \doteq - \beta_{6}/(2 \beta_{7}) =0.329$. Comparing ranges and the stationary points, we conclude that in our data, the stationary point of each rating entropy variable lies within its observed data range. In fact, stationary point for each of these variables is also evident in Figures \ref{fig: marginsplot_ind_ent_self}, \ref{fig: marginsplot_ind_ent_others} and \ref{fig: marginsplot_mkt_entmkt}. This provides further validation for the inverted U-shaped relationship we find in Sections \ref{Sec: Installer-level} and \ref{Sec: Market-level}.
	
	
	We used a common criteria to identify inverted U-shaped relationships.  Some researcher argue that for the inverted U-shaped relationship to be meaningful for an explanatory variable, the stationary point for that variable should not be too close to the end points of the data range or too far from the sample mean (e.g., 3 standard deviation far) \citep{lind2010or}. This concern does not apply to our analysis as the stationary point for each rating entropy is close to its sample mean. Specifically, $S_{\text{self}}$, $S_{\text{others}}$ and $S_{\text{mkt}}$ are respectively $1.25$, $0.51$ and $1.39$ standard deviation away from their sample means.
	
	
	
	
	
	
	%
	%The commonly used criterion for identifying
	%an inverted-U-shaped relationship, i.e., the significance
	%of the quadratic term that we used in the main analysis,
	%has been questioned in some recent literature
	%(Lind and Mehlum 2010). This literature argues that
	%the quadratic specification may erroneously create an
	%extreme point even though the true relationship is
	%concave and monotone. We believe that this concern
	%does not necessarily apply to our analysis because
	%our extreme points are close to the sample means.
	
	
	%stationary point should lie within the range of the variable
	%in our sample.
	%To distinguish an increasing concave relationship
	%from an inverted U-shaped relationship,
	%we compute the stationary points for temporary
	%and part-time labor mixes as −1/22 and −3/24
	%,
	%respectively, for the sales equation and check if
	%they lie within the sample. We repeat the process
	%for the expense and profitability equations to test
	%for U-shaped and inverted U-shaped relationships,
	%respectively.
	%
	%Aiken and West (1991) state that
	%to identify a nonmonotonic relationship, the stationary
	%point should lie within the meaningful range
	%of the variable. To test if the stationary point lies
	%in a meaningful range, they suggest computing the
	%slope of the curve for different points of the variable
	%and ensuring that the slope is significantly different
	%from zero and of different signs on either side
	%of the stationary point
	%
	%For example, in the sales
	%equation, the slope of the curve is given by 1 +
	%22TempMixit, and the standard error is calculated
	%as p
	%‘11 + 4TempMixit‘12 + 4‘224TempMixit5
	%2
	%. Here ‘11
	%and ‘22 are the variance of 1 and 2
	%, respectively, and
	%‘12 is the covariance between 1 and 2
	%. Table 4(a)
	%shows the tests of the simple slopes for temporary
	%and part-time labor-mix variables at the stationary
	%point, and ±1 SD, minimum, and maximum values
	%in the sample. Since the simple slopes on either side
	%of the stationary point are statistically significant and
	%are of different signs for both temporary and parttime
	%labor-mix variables, we can conclude that the
	%inverted U-shaped relationship is supported within
	%the sample for the sales equation. We repeat the similar
	%analysis for the expense equation, and the results
	%are reported in Table 4(b). We find that our statistical
	%tests confirm the U-shaped relationship for temporary
	%labor-mix variable. Since the stationary point
	
	
	
	
	\subsection{Alternative Test for Inverted U-Shaped Relationship: Spline Regression}
	
	
	Up to this section, we identified an inverted U-shaped relationship between entropy measures and dependent variables by applying a standard technique. That is, by running a polynomial regression, showing the significance of linear and quadratic terms of entropy measure, and identifying the positive sign for the linear term and the negative sign for the quadratic term. See, e.g., \cite{tan2014does} and \cite{kesavan2014volume} that apply this technique. For robustness check, we also perform spline regressions on both individual and market level analysis to identify the non-monotone effects via a different approach.

% This robustness check is also standard in the literature \citep{kesavan2014volume}.
	
	Spline regressions use breakpoints (\emph{knots}) to capture the changes in coefficients for different intervals of explanatory variables. We perform spline regressions with 1 knot and with 2 knots for entropy variables. Knots divide the range of the explanatory variable of interest into sub-ranges. For each of these sub-ranges, we create a spline variable of the considered explanatory variable, and allow for a different linear relationship between the response variable and the spline variable. For example, for 1 knot, we create two spline variables $\text{Rating\_Entropy\_Others\_1}$ and  $\text{Rating\_Entropy\_Others\_2}$, and consider the linear term of either one only in one of the two ranges of the variable. For the installer-level analysis, we plug in the linear term of $\text{Rating\_Entropy\_Others\_1}$ (respectively, $\text{Rating\_Entropy\_Others\_2}$) in place of the combination of linear and quadratic terms of $\text{Rating\_Entropy\_Others}$ in \eqref{model_ind_3} when $\text{Rating\_Entropy\_Others}$ is smaller (respectively, larger) than the breakpoint.  Likewise, in separate regressions, we repeat this procedure for $\text{Rating\_Entropy\_Self}$ in the installer-level analysis (based on \eqref{model_ind_3}) and  for $\text{Rating\_Entropy\_Mkt}$ in the market-level analysis (based on \eqref{reg: market-level-rating}). We also further extend this alternative testing to consider spline regressions with 2 knots, which require us to create three spline variables for each rating entropy measure.
	
	%By dividing the sample into subsamples through different thresholds or knots and fitting polynomial regression in each subsample.
	
	
	The results are presented in Tables \ref{rob_spline_ind} and \ref{rob_spline_mkt}. In Table \ref{rob_spline_ind}, we report the spline regression estimates with one knot on $\text{Rating\_Entropy\_Others}$ in column (I) and with one knot on $\text{Rating\_Entropy\_Self}$ in column (II). We find that the coefficient of the first spline is positive and significant while the second one (which is valid above the breakpoint) is negative and significant ($p<0.001)$, supporting the inverted U-shaped relationships between either rating entropy measure and the installer's activity level. The conclusions are similar when we consider the case with 2 knots as shown in column (III) and (IV) of Table \ref{rob_spline_ind}.  Next, we consider the market-level analysis with results presented in Table \ref{rob_spline_mkt}. For the case with 1 breakpoint for Rating\_Entropy\_Mkt, first positive and then negative and significant ($p<0.001$) coefficients associated with the two splines in column (I) further validate the inverted U-shaped relationship we established on the market-level. We also find that when we move to the case with two breakpoints, non-monotone relationship is still preserved in the market level.
	
	
	%\subsection{Excluding Inactive Installers}
	%
	% We now run a robustness check by excluding installers that have been inactive (i.e., made 0 proposals) for the last two months in the online marketplace. With this modification, our regression results are presented in Table \ref{rob_exclude_inactive}.  The results are consistent, especially on the inverted U-shaped relationship between rating entropy self and an installer's activity level.
	%
	%% The first two columns are results excluding these said installers ( cluster standard errors on market level - column (1); individual level - column (2)) .
	
	\subsection{Alternative Approach to Measure Text-based Dispersion} \label{Sec: Mean}
	
	In Section \ref{Sec: TextMining}, we measured the text-based dispersion by taking the median of cosine distances. Alternatively, one can consider the mean of cosine distances to create the text-based dispersion variables. Tables \ref{reg_ind_rob_text_mean} and \ref{reg_mkt_rob_text_mean} report the estimation results when the mean (rather than median) of cosine distances is considered. By Table \ref{reg_ind_rob_text_mean}, the installer-level text-based dispersion result in Section \ref{Sec: TextMining} continues to hold. In addition, Table \ref{reg_mkt_rob_text_mean} suggests that when the mean (rather than median) of cosine distances is considered, the market-level text-based review dispersion has a significant and inverted U-shaped impact on the market transaction.
	
	
	
	%\subsection{Market Level Alternative Measure of Success}
	%
	%In the analysis of ratings dispersion on local market level performance, we used total quotes accepted by consumers to measure the success of marketplace. We present results using total quotes given out by installers, and it remains consistent, as table \ref{reg_mkt_alt_measure} shows.
	
	
	
	
	
\section{Conclusions}

Our paper contributes to the literature by empirically investigating if and how the dispersion in customer reviews impacts a firm's activity level (i.e., logged number of proposals) and the number of matches in an online marketplace where firms are active. To the best of our knowledge, there is no prior work that examines this topic. Our findings offer key insights to a marketplace operator.

We find that there is a significant and inverted U-shaped relationship between a firm's review dispersion and its activity level in the online marketplace. Thus, an increase in a firm's review dispersion can increase or decrease its activity level in the marketplace, depending on the level of dispersion. If the mentioned dispersion is below (respectively, above) a certain threshold, an increase in that dispersion increases (respectively, decreases) the firm's activity level in the marketplace. Furthermore, we find that a firm's activity level  has a significant and inverted U-shaped relationship with competitors' rating dispersion in the online marketplace. Thus, similar to the previous finding, an increase in this type of dispersion can encourage or deter the firm from making a proposal, depending on the dispersion level.

We identify a significant and inverted U-shaped relationship between the market transactions and the review dispersion at a local market level. This finding has a key implication for an online marketplace operator: Having all sellers with 5 stars might not be favorable to the marketplace operator. Review dispersion up to a particular level can help an online marketplace operator in terms of number of matches.


Regarding the methodology, to analyze text reviews, we incorporated two text-mining methods: VADER that assigns a one-dimensional sentiment intensity score to each text review  and BERT that converts each piece of text review to a numerical vector via deep learning. We used the latter to measure the content dissimilarity among text reviews with precision. To our knowledge, our paper is the first that uses the deep-learning based advanced text-mining method BERT in the operations management literature.  Apart from this, our paper provides a showcase for the state-of-the-art clustering method OPTICS that has many advantages over common clustering techniques. This advanced clustering method has not been used in the OM literature yet. Overall, these methods have the potential to facilitate research in various contexts in the operations management literature.
	
%	{
%%\vspace{-5pt}
%\setlength{\bibsep}{6.5pt}
%\SingleSpacedXI
%%\bibliographystyle{ormsv080} % outcomment this and next line in Case 1
%\bibliographystyle{agsm} % outcomment this and next line in Case 1
%\bibliography{reference_sfe} % if more than one, comma separated}
%}

{
%\setlength{\bibsep}{6.5pt}
\SingleSpacedXI
%\bibliographystyle{informs2014} % outcomment this and next line in Case 1
\bibliographystyle{informs2014} % outcomment this and next line in Case 1
\bibliography{solarlits} % if more than one, comma separated
%\OneAndAHalfSpacedXI
}	


%\newpage
\ECSwitch

%\ECDisclaimer
%%%%%%%%%%%%%%%%%%%%%%%%%%%%%%%%%%%%%%%%%%%%%%%%%%%%%%%%%%

%%% Main head for the e-companion
\vspace{-10pt}
\ECHead{Electronic Companion for ``Do Noisy Customer Reviews Discourage Platform Sellers? Empirical and Textual Analysis of an Online Solar Marketplace with Deep Learning''}
%\begin{APPENDICES}
% \label{vif_ind}
%% Please add the following required packages to your document preamble:
% \usepackage{booktabs}
\begin{table}[]
\centering
\begin{tabular}{@{}lll@{}}
\toprule
Variable                           & VIF  & 1/VIF \\ \midrule
Rating\_Entropy\_Self                & 7.10 & 0.14  \\
Rating\_Entropy\_Self$^2$            & 5.81 & 0.17  \\
Rating\_Entropy\_Others              & 1.83 & 0.55  \\
Rating\_Entropy\_Others$^2$          & 1.46 & 0.68  \\
Average\_Rating\_Self                & 4.59 & 0.22  \\
Average\_Rating\_Others             & 1.40 & 0.71  \\
Reviews\_Count                      & 1.24 & 0.81  \\
Experience                         & 1.46 & 0.68  \\
Price\_Difference                  & 1.03 & 0.97  \\
Market\_LogRevenue                 & 1.65 & 0.61  \\ \bottomrule
\end{tabular}
\caption{VIF table: Installer Level}
\label{vif_ind}
\end{table}

%\begin{table}[H]
%\centering
%\begin{tabular}{@{}lll@{}}
%\toprule
%Variable                           & VIF  & 1/VIF \\ \midrule
%Rating\_Entropy\_Self                & 7.10 & 0.14  \\
%Rating\_Entropy\_Self$^2$            & 5.81 & 0.17  \\
%Rating\_Entropy\_Others              & 1.83 & 0.55  \\
%Rating\_Entropy\_Others$^2$          & 1.46 & 0.68  \\
%Average\_Rating\_Self                & 4.59 & 0.22  \\
%Average\_Rating\_Others             & 1.40 & 0.71  \\
%Reviews\_Count                      & 1.24 & 0.81  \\
%Experience                         & 1.46 & 0.68  \\
%Price\_Difference                  & 1.03 & 0.97  \\
%Market\_LogRevenue                 & 1.65 & 0.61  \\ \bottomrule
%\end{tabular}
%\caption{Variance Inflation Factors (VIFs) for the Installer-Level Regression in Section \ref{Sec: Installer-level}}
%\label{vif_ind}
%\end{table}
%
%%% Please add the following required packages to your document preamble:
% \usepackage{booktabs}
\begin{table}[H]
\centering
\begin{tabular}{@{}lcc@{}}
\toprule
Variable           & VIF  & 1/VIF    \\ \midrule
Rating\_Entropy\_Mkt    & 6.67 & 0.149965 \\
Rating\_Entropy\_Mkt$^2$ & 6.21 & 0.160950 \\
Average\_Rating\_Mkt     & 1.58 & 0.632615 \\
Experience     & 1.46 & 0.685113 \\
Price\_Difference   & 1.01 & 0.986885 \\
Market\_LogRevenue     & 1.45 & 0.690279 \\ \bottomrule
\end{tabular}
\caption{VIF Table:Market level}
\label{vif_mkt}
\end{table} 
%\begin{table}[H]
%\centering
%\begin{tabular}{@{}lcc@{}}
%\toprule
%Variable           & VIF  & 1/VIF    \\ \midrule
%Rating\_Entropy\_Mkt    & 6.67 & 0.149965 \\
%Rating\_Entropy\_Mkt$^2$ & 6.21 & 0.160950 \\
%Average\_Rating\_Mkt     & 1.58 & 0.632615 \\
%Experience     & 1.46 & 0.685113 \\
%Price\_Difference   & 1.01 & 0.986885 \\
%Market\_LogRevenue     & 1.45 & 0.690279 \\ \bottomrule
%\end{tabular}
%\caption{Variance Inflation Factors (VIFs) for the Market-Level Regression in Section \ref{Sec: Market-level}}
%\label{vif_mkt}
%\end{table}




	% Please add the following required packages to your document preamble:
% \usepackage{booktabs}
% \usepackage{graphicx}
\begin{table}[H]
\centering
\begin{tabular}{@{}lllllllll@{}}
\toprule
Variables                     & (1)    & (2)    & (3)    & (4)    & (5)    & (6)    & (7)   & (8) \\ \midrule
(1) Average\_Sentiment\_Self  & 1      &        &        &        &        &        &       &     \\
(2) Average\_Rating\_Self     & 0.832  & 1      &        &        &        &        &       &     \\
(3) Average\_Sentiment\_Other & -0.036 & -0.074 & 1      &        &        &        &       &     \\
(4) Average\_Rating\_Other    & -0.073 & -0.023 & 0.434  & 1      &        &        &       &     \\
(5) Text\_Dispersion\_Other   & -0.064 & 0.005  & -0.032 & -0.076 & 1      &        &       &     \\
(6) Rating\_Entropy\_Other    & 0.122  & 0.07   & -0.089 & -0.527 & 0.081  & 1      &       &     \\
(7) Text\_Dispersion\_Self    & 0.1    & 0.136  & -0.089 & -0.003 & -0.091 & 0.009  & 1     &     \\
(8) Rating\_Entropy\_Self     & -0.081 & -0.086 & 0.064  & 0.023  & 0.026  & -0.089 & 0.304 & 1   \\ \bottomrule
\end{tabular}%
\caption{Correlation of Ratings and Text-based Measures (Installer-level) }
\label{corr_measures_dispersion_text_and_rating}
\end{table} 
	% Please add the following required packages to your document preamble:
% \usepackage{booktabs}
% \usepackage{graphicx}
\begin{table}[H]
\centering
\begin{tabular}{@{}llllllllll@{}}
\toprule
 & Variables                       & (1)    & (2)    & (3)    & (4)    & (5)    & (6)    & (7)       & (8)      \\ \midrule
 & (1) Average\_Sentiment\_Self    & 1.000  & \multicolumn{7}{l}{}                                              \\
 & (2) Average\_Sentiment\_Others  & -0.024 & 1.000  & \multicolumn{6}{l}{}                                     \\
 & (3) Text-based\_Entropy\_Others & -0.063 & -0.055 & 1.000  & \multicolumn{5}{l}{}                            \\
 & (4) Text-based\_Entropy\_Self   & 0.106  & -0.022 & -0.092 & 1.000  & \multicolumn{4}{l}{}                   \\
 & (5) Review\_Count               & 0.166  & 0.070  & -0.027 & 0.241  & 1.000  & \multicolumn{3}{l}{}          \\
 & (6) Experience                  & 0.082  & 0.205  & -0.064 & 0.082  & 0.124  & 1.000  & \multicolumn{2}{l}{} \\
 & (7) Price\_Difference           & -0.039 & 0.005  & -0.009 & -0.048 & -0.027 & -0.033 & 1.000     &          \\
 & (8) Market\_LogRevenue          & 0.034  & 0.246  & -0.127 & 0.003  & -0.052 & 0.553  & -0.062    & 1.000    \\ \bottomrule
\end{tabular}%
\caption{Correlation Matrix -  Individual Level Text-based Analysis }
\label{corr_ind_text}
\end{table} 
	% Please add the following required packages to your document preamble:
% \usepackage{booktabs}
% \usepackage{graphicx}
\begin{table}[H]
\centering
\begin{tabular}{lcccccccc} \toprule
Variables                   & (1)    & (2)    & (3)   & (4)    & (5)    & (6)        & (7)     \\\midrule
(1) Text\_Dispersion\_Mkt   & 1      & \multicolumn{6}{l}{}                                    \\
(2) Rating\_Entropy\_Mkt    & 0.068  & 1      & \multicolumn{5}{l}{}                           \\
(3) Average\_Sentiment\_Mkt & -0.43  & 0.118  & 1     & \multicolumn{4}{l}{}                   \\
(4) Average\_Rating\_Mkt    & -0.097 & -0.624 & 0.173 & 1      & \multicolumn{3}{l}{}          \\
(5) Average\_Experience              & -0.113 & 0.198  & 0.053 & -0.107 & 1      & \multicolumn{2}{l}{} \\
(6) Price\_Difference\_Mkt        & 0.127  & -0.018 & 0.017 & -0.041 & -0.012 & 1          &         \\
(7) Market\_LogRevenue      & -0.087 & 0.157  & 0.23  & -0.041 & 0.498  & -0.126     & 1 \\\bottomrule
\end{tabular}%
\caption{Correlation Matrix -  Market Level Text-based Analysis}
\label{ corr_mkt_text}
\end{table} 
	% Please add the following required packages to your document preamble:
% \usepackage{booktabs}
% \usepackage{graphicx}
\begin{table}[H]
\centering
\begin{threeparttable}[t]
\begin{tabular}{@{}lcccc@{}}
\toprule
                            & (I)         & (II)        & (III)       & (IV)        \\
                            & Installer's & Installer's & Installer's & Installer's \\
Variables                   & Activity Level & Activity Level & Activity Level & Activity Level \\ \midrule
Text-based\_Entropy\_Self   & 5.326***    & 5.389***    & 5.014***    & 4.890***    \\
                            & (0.000)     & (0.000)     & (0.000)     & (0.000)     \\
Text-based\_Entropy\_Self$^2$    & -23.71***           & -23.54***              & -20.29***      & -20.40***      \\
                            & (0.000)     & (0.000)     & (0.000)     & (0.000)     \\
Text-based\_Entropy\_Others & -1.288      & -0.989      & -1.640      & -1.955      \\
                            & (0.389)     & (0.509)     & (0.275)     & (0.193)     \\
Text-based\_Entropy\_Others$^2$   & -9.593              & -13.13                 & 1.348          & 5.132          \\
                            & (0.623)     & (0.517)     & (0.944)     & (0.782)     \\
Average\_Rating\_Self       & -0.942***   &             &             & -0.998***   \\
                            & (0.000)     &             &             & (0.000)     \\
Average\_Rating\_Others     & 0.000219    &             &             & -0.00483    \\
                            & (0.991)     &             &             & (0.813)     \\
Average\_Sentiment\_Self    &             & -0.431      & -0.404      &             \\
                            &             & (0.090)     & (0.129)     &             \\
Average\_Sentiment\_Others  &             & 0.111       & 0.0806      &             \\
                            &             & (0.378)     & (0.524)     &             \\
Rating\_Entropy\_Self       &             &             & 2.044***    & 2.161***    \\
                            &             &             & (0.000)     & (0.000)     \\
Rating\_Entropy\_Self$^2$              &                     &                        & -4.246***      & -4.344***      \\
                            &             &             & (0.000)     & (0.000)     \\
Rating\_Entropy\_Others     &             &             & 0.399*      & 0.384*      \\
                            &             &             & (0.033)     & (0.045)     \\
Rating\_Entropy\_Others$^2$          &                     &                        & -2.382***      & -2.376***      \\
                            &             &             & (0.000)     & (0.000)     \\
Review\_Count               & 0.0489***   & 0.0492***   & 0.0420***   & 0.0413***   \\
                            & (0.000)     & (0.000)     & (0.000)     & (0.000)     \\
Experience                  & 0.178***    & 0.175***    & 0.176***    & 0.177***    \\
                            & (0.001)     & (0.001)     & (0.001)     & (0.001)     \\
Price\_Difference           & 0.0952      & 0.101       & 0.125       & 0.118       \\
                            & (0.278)     & (0.247)     & (0.158)     & (0.181)     \\
Market\_LogRevenue          & -0.0164***  & -0.0165***  & -0.0159***  & -0.0158***  \\
                            & (0.000)     & (0.000)     & (0.001)     & (0.001)     \\
Constant                    & 2.374***    & 2.075***    & 2.379***    & 2.649***    \\
                            & (0.000)     & (0.000)     & (0.000)     & (0.000)     \\
Fixed Effect                & Yes        & Yes         & Yes          &Yes \\
State Dummies               & Yes        & Yes        & Yes           &Yes\\                          
Observations                & 4562        & 4562        & 4562        & 4562        \\
Adjusted-R$^2$                          & 0.633       & 0.633       & 0.638       & 0.638       \\
AIC                         & 13202.7     & 13200.9     & 13147.9     & 13147.6     \\
BIC                         & 13292.7     & 13290.8     & 13263.5     & 13263.2     \\ \bottomrule
\end{tabular}%
\begin{tablenotes}
\item Note: $p$-value in parentheses; $^\star p<0.05;^{\star\star} p<0.01;^{\star\star\star} p<0.001$
\end{tablenotes}
\end{threeparttable}
\caption{Installer Level Analysis with Variables Derived from Text Analysis}
\label{reg_ind_withtext}
\end{table} 
	% Please add the following required packages to your document preamble:
% \usepackage{booktabs}
% \usepackage{graphicx}
\begin{table}
\centering
\begin{threeparttable}
\begin{tabular}{@{}lccc@{}}
\toprule
                                               & (I)            & (II)           & (III)          \\
                                               & Market    & Market    & Market    \\
Variables                                      & Transaction & Transaction & Transaction \\ \midrule
Rating\_Entropy\_Mkt                         &                       &                       & 1.605***              \\
                                             &                       &                       & (0.000)               \\
Rating\_Entropy\_Mkt$^2$ &                       &                       & -1.622***             \\
                                             &                       &                       & (0.000)               \\
Average\_Rating\_Mkt                         & -0.259                &                       &                       \\
                                             & (0.092)               &                       &                       \\
Average\_Sentiment\_Mkt                      &                       & -1.316**              & -1.180*               \\
                                             &                       & (0.006)               & (0.012)               \\
Average\_Experience                          & 0.0193*               & 0.0148                & 0.00983               \\
                                             & (0.016)               & (0.058)               & (0.196)               \\
Price\_Difference\_Mkt                       & 0.228                 & 0.228                 & 0.292                 \\
                                             & (0.314)               & (0.301)               & (0.182)               \\
Market\_LogRevenue                           & -0.0820               & -0.0652               & -0.0517               \\
                                             & (0.066)               & (0.125)               & (0.214)               \\
Text\_Dispersion\_Mkt                        & 3.014                 & 1.318                 & -1.856                \\
                                             & (0.443)               & (0.736)               & (0.632)               \\
Text\_Dispersion\_Mkt$^2$                    & -7.258                & -3.138                & 3.539                 \\
                                             & (0.420)               & (0.730)               & (0.691)               \\
Constant                                     & 3.124*                & 2.843*                & 2.688*                \\
                                             & (0.024)               & (0.017)               & (0.021)               \\
Fixed Effect                                   & Yes            & Yes            & Yes            \\
State Dummies                                  & Yes            & Yes            & Yes            \\
Observations                                 & 642          & 642          & 642                  \\
Adjusted R$^2$                               & 0.739                 & 0.743                 & 0.750                 \\
AIC                                            & 8853.4         & 8849.3         & 8843.7         \\
BIC                                            & 8925.9         & 8933.9         & 8928.3         \\ \bottomrule
\end{tabular}%
\begin{tablenotes}
\item Note: $p$-value in parentheses; $^\star p<0.05;^{\star\star} p<0.01;^{\star\star\star} p<0.001$
\end{tablenotes}
\end{threeparttable}
\caption{Market Level Analysis with Variables Derived from Text Analysis}
\label{reg_mkt_textbased}
\end{table} 
		% Please add the following required packages to your document preamble:
% \usepackage{booktabs}
\begin{table}[H]
\centering
\begin{tabular}{@{}ccc@{}}
\toprule
\multicolumn{1}{l}{\textbf{Installer Level Dynamic Panel}} & \multicolumn{1}{l}{} & \multicolumn{1}{l}{} \\ \midrule
Order                                   & z                    & $p-value$  \\
$H_{0}$: No correlation between $\Delta_{i,t}$ and $\Delta_{i,t-1}$ & -9.8283  & 0.00 \\
$H_{0}$: No correlation between $\Delta_{i,t}$ and $\Delta_{i,t-2}$  & -0.66053 & 0.5089 \\
 Sargan Test for overidentifying restriction & $\chi^2(1525)$  =  238.2181& 1.000 \\ \midrule
\textbf{Market Level Dynamic Panel}          &    &      \\ \midrule
Order                           & z      & $p-value$  \\
$H_{0}$: No correlation between $\Delta_{i,t}$ and $\Delta_{i,t-1}$  & -2.7882 & 0.01 \\
$H_{0}$: No correlation between $\Delta_{i,t}$ and $\Delta_{i,t-2}$  & 0.04295  & 0.9657 \\
 Sargan Test for overidentifying restriction & $\chi^2(544)$ =  22.77427 & 1.000 \\ \bottomrule
\end{tabular}
\caption{Dynamic Panel Specification Checks
}
\label{autocorrelation_test}
\end{table} 
		% Please add the following required packages to your document preamble:
% \usepackage{booktabs}
% \usepackage{graphicx}
\begin{table}[H]
\centering
\begin{tabular}{@{}lcccc@{}}
	\toprule
	& (I)            & (II)           & (III)          & (IV)           \\
	& Installer's    & Installer's    & Installer's    & Installer's    \\
	Variables                                & Activity Level & Activity Level & Activity Level & Activity Level \\ \midrule
	Installer\_Activity$_t$                 & 0.510***       & 0.507***       & 0.509***       & 0.502***       \\
	& (0.000)        & (0.000)        & (0.000)        & (0.000)        \\
	Installer\_Activity$_{t-1}$       & 0.0393         & 0.0374         & 0.0363         & 0.0369         \\
	& (0.130)        & (0.155)        & (0.176)        & (0.158)        \\
	Rating\_Entropy\_Self                    &                &                & 1.322**        & 1.351**        \\
	&                &                & (0.006)        & (0.004)        \\
	Rating\_Entropy\_Self$^2$ &                &                & -1.578*        & -1.740*        \\
	&                &                & (0.043)        & (0.026)        \\
	Rating\_Entropy\_Others           &                & 0.733          &                & 0.695          \\
	&                & (0.059)        &                & (0.069)        \\
	Rating\_Entropy\_Others$^2$       &                & -1.300*        &                & -1.253*        \\
	&                & (0.041)        &                & (0.031)        \\
	Average\_Rating\_Self                    & -0.0618***     & -0.0713***     & -0.0662***     & -0.0680***     \\
	& (0.000)        & (0.000)        & (0.000)        & (0.000)        \\
	Average\_Rating\_Others                  & -0.180         & -0.191         & -0.137         & -0.162         \\
	& (0.128)        & (0.138)        & (0.232)        & (0.183)        \\
	Review\_Count                            & 0.0165*        & 0.0175**       & 0.0134*        & 0.0105         \\
	& (0.023)        & (0.007)        & (0.024)        & (0.077)        \\
	Experience                               & -0.0554        & -0.0445        & -0.0564        & -0.0528        \\
	& (0.333)        & (0.424)        & (0.304)        & (0.326)        \\
	Price\_Difference                        & -0.0164        & 0.110          & 0.0736         & 0.0987         \\
	& (0.879)        & (0.324)        & (0.457)        & (0.366)        \\
	Market\_LogRevenue                       & 0.00348        & 0.00349        & 0.00367        & 0.00283        \\
	& (0.606)        & (0.618)        & (0.593)        & (0.681)        \\
	Observations                             & 3757           & 3757           & 3757           & 3757           \\ \bottomrule
\end{tabular}%
\begin{tablenotes}
\item Note: $p$-value in parentheses; $^\star p<0.05;^{\star\star} p<0.01;^{\star\star\star} p<0.001$
\end{tablenotes}
\vspace{10pt}
\caption{Robustness Check - Installer Level Dynamic Panels}
\label{rob_ind_dynamic}
\end{table} 
		% Please add the following required packages to your document preamble:
% \usepackage{booktabs}
% \usepackage{graphicx}
\begin{table}[H]
\centering
\begin{threeparttable}[t]
\begin{tabular}{@{}lcc@{}}
\toprule
                         & (I)            & (II)           \\
                         & Market Transaction       & Market Transaction       \\
Variables                &   &   \\ \midrule
Rating\_Entropy\_Mkt     & 1.501***       & 0.751*         \\
                         & (0.000)        & (0.016)        \\
Rating\_Entropy\_Mkt$^2$ & -2.456***      & -1.609***      \\
                         & (0.000)        & (0.000)        \\
Average\_Rating\_Mkt     & -0.207         & -0.141         \\
                         & (0.169)        & (0.476)        \\
Average\_Experience      & 0.0102         & -0.00118       \\
                         & (0.157)        & (0.882)        \\
Price\_Difference\_Mkt   & 0.0349         & -0.313         \\
                         & (0.865)        & (0.089)        \\
Market\_LogRevenue       & -0.0264        & 0.0403         \\
                         & (0.502)        & (0.051)        \\
Market\_Transaction$_{t}$&                & 0.0238         \\
                         &                & (0.685)        \\
Market\_Transaction$_{t-1}$    &                & -0.00533       \\
                         &                & (0.917)        \\
Market\_Transaction$_{t-2}$   &                & 0.198**        \\
                         &                & (0.001)        \\
Market Fixed Effect      & Yes            & No             \\
Weighted State Dummies   & Yes            & Yes            \\
%Constant                 & 2.176**        &                \\
%                         & (0.003)        &                \\
Observations             & 642            & 421            \\ \bottomrule
\end{tabular}%
\begin{tablenotes}
\item Note: $p$-value in parentheses; $^\star p<0.05;^{\star\star} p<0.01;^{\star\star\star} p<0.001$
\end{tablenotes}
\end{threeparttable}
\vspace{10pt}
\caption{Robustness Check - Market Level Dynamic Panels}
\label{rob_mkt_dynamic}
\end{table} 

		% Please add the following required packages to your document preamble:
% \usepackage{booktabs}
% \usepackage{graphicx}
\begin{table}[]
\centering
\begin{threeparttable}[t]
\begin{tabular}{@{}lcccc@{}}
\toprule
                           & (I)            & (II)           & (III)          & (IV)           \\ 
                           & Installer's    & Installer's    & Installer's    & Installer's    \\
Variables                  & Activity Level & Activity Level & Activity Level & Activity Level \\ \midrule
Rating\_Entropy\_Others\_1 & 0.512*    &           &          &          \\
                           & (0.013)   &           &          &          \\
Rating\_Entropy\_Others\_2 & -1.934*** &           &          &          \\
                           & (0.000)   &           &          &          \\
Rating\_Entropy\_Self\_1   &           & 1.200***  &          &          \\
                           &           & (0.000)   &          &          \\
Rating\_Entropy\_Self\_2   &           & -3.072*** &          &          \\
                           &           & (0.000)   &          &          \\
Rating\_Entropy\_Others\_1 &           &           & 0.834*** &          \\
                           &           &           & (0.001)  &          \\
Rating\_Entropy\_Others\_2 &           &           & -1.164*  &          \\
                           &           &           & (0.017)  &          \\
Rating\_Entropy\_Others\_3 &           &           & -1.763*  &          \\
                           &           &           & (0.020)  &          \\
Rating\_Entropy\_Self\_1   &           &           &          & 1.746*** \\
                           &           &           &          & (0.000)  \\
Rating\_Entropy\_Self\_2   &           &           &          & -1.015   \\
                           &           &           &          & (0.099)  \\
Rating\_Entropy\_Self\_3   &           &           &          & -4.641** \\
                           &           &           &          & (0.008)  \\
Observations               & 4562      & 4562      & 4562     & 4562     \\
Adjusted R$^2$                         & 0.631     & 0.633     & 0.632    & 0.633    \\
AIC                       & 13222.5   & 13204.4   & 13218.4  & 13203.5  \\
BIC                        & 13306.0   & 13287.9   & 13308.3  & 13293.4 \\ \bottomrule
\end{tabular}%
\begin{tablenotes}
\item Note: $p$-value in parentheses; $^\star p<0.05;^{\star\star} p<0.01;^{\star\star\star} p<0.001 $
\end{tablenotes}
\end{threeparttable}
\caption{Alternative Inverted-U Testing: Spline Regressions (Installer Level)}
\label{rob_spline_ind}
\end{table}
		% Please add the following required packages to your document preamble:
% \usepackage{booktabs}
% \usepackage{graphicx}
\begin{table}[]
\centering
 \begin{threeparttable}[t]
\begin{tabular}{@{}lcc@{}}
\toprule
                        & (I)            & (II)           \\ 
                        & Market Transaction       & Market Transaction       \\
Variables               &   &   \\ \midrule
Rating\_Entropy\_Mkt\_1 & 1.197***       &                \\
                        & (0.000)        &                \\
Rating\_Entropy\_Mkt\_2 & -1.144**       &                \\
                        & (0.008)        &                \\
Rating\_Entropy\_Mkt\_1 &                & 2.157***       \\
                        &                & (0.000)        \\
Rating\_Entropy\_Mkt\_2 &                & -2.100***      \\
                        &                & (0.000)        \\
Rating\_Entropy\_Mkt\_3 &                & 0.356          \\
                        &                & (0.606)        \\
Observations            & 642            & 642            \\
Adjusted R$^2$             & 0.720          & 0.732          \\
AIC                     & 1101.0         & 1074.3         \\
BIC                     & 1136.7         & 1114.4         \\ \bottomrule
\end{tabular}%
\begin{tablenotes}
\item Note: $p$-value in parentheses; $^\star p<0.05;^{\star\star} p<0.01;^{\star\star\star} p<0.001 $
\end{tablenotes}
\end{threeparttable}
\caption{Alternative Inverted-U Testing: Spline Regressions (Market Level)}
\label{rob_spline_mkt}
\end{table}
%		\input{reg_mkt_alt_measure.tex}
		% Please add the following required packages to your document preamble:
% \usepackage{booktabs}
\begin{table}[]
\centering
\begin{threeparttable}
\begin{tabular}{@{}lccc@{}}
\toprule
                  & (I)      & (II)    & (III)   \\
& Installer's Activity & Installer's Activity & Installer's Activity \\
Variables & Level    & Level   & Level   \\ \midrule
Text\_Dispersion\_Self                                 &            & 11.42*     & 13.28*     \\
                                                           &            & (0.032)    & (0.016)    \\
Text\_Dispersion\_Self$^2$    &            & -34.86*    & -37.87*    \\
                                                           &            & (0.019)    & (0.013)    \\
Text\_Dispersion\_Other                                & 14.22      & 13.10      & 14.26      \\
                                                           & (0.086)    & (0.113)    & (0.080)    \\
Text\_Dispersion\_Other$^2$  & -59.45*    & -54.83     & -57.21*    \\
                                                           & (0.035)    & (0.051)    & (0.039)    \\
Average\_Rating\_Self                                             & -0.436     & -0.489     &            \\
                                                           & (0.086)    & (0.062)    &            \\
Average\_Rating\_Others                                    & -0.0164    & -0.0166    &            \\
                                                           & (0.468)    & (0.466)    &            \\
Average\_Sentiment\_Self                                   &            &            & 0.268      \\
                                                           &            &            & (0.438)    \\
Average\_Sentiment\_Others                                 &            &            & 0.0770     \\
                                                           &            &            & (0.802)    \\
Reviews\_Count                                              & 0.0472*    & 0.0450*    & 0.0466*    \\
                                                           & (0.000)    & (0.000)    & (0.000)    \\
Experience                                                 & 0.104      & 0.0946     & 0.100      \\
                                                           & (0.124)    & (0.164)    & (0.144)    \\
Price\_Differences                                         & -0.00831   & 0.0141     & 0.0249     \\
                                                           & (0.936)    & (0.893)    & (0.814)    \\
Market\_LogRevenue                                         & -0.0162*   & -0.0158*   & -0.0156*   \\
                                                           & (0.002)    & (0.002)    & (0.002)    \\
Constant                                                   & 1.507*     & 0.747      & 0.236      \\
                                                           & (0.029)    & (0.348)    & (0.782)    \\
State Dummies  & Yes        & Yes        & Yes        \\
Fixed Effects  & Yes        & Yes        & Yes        \\
Adjusted R$^2$                                             & 0.668      & 0.669      & 0.670      \\
AIC                                                        & 8853.4     & 8849.3     & 8843.7     \\
BIC                                                        & 8925.9     & 8933.9     & 8928.3     \\ \bottomrule
\end{tabular}
\begin{tablenotes}
\item Note: $p$-value in parentheses; $^\star p<0.05;^{\star\star} p<0.01;^{\star\star\star} p<0.001 $
\item The text-based dispersion variables are derived by taking the mean, instead of median, of the pairwise cosine distances.
\end{tablenotes}
\end{threeparttable}
\vspace{10pt}
\caption{Robustness Check - Installer Level Analysis with Text-based Dispersion from Mean Cosine Distances}
\label{reg_ind_rob_text_mean}
\end{table} 
        % Please add the following required packages to your document preamble:
% \usepackage{booktabs}
% \usepackage{graphicx}
\begin{table}[]
\centering
\begin{threeparttable}[t]
\begin{tabular}{@{}lccc@{}}
\toprule
                                                 & (I)                & (II)               & (III)              \\
                                                 & Market Transaction & Market Transaction & Market Transaction \\\midrule
Rating\_Entropy\_Mkt                             &                    &                    & 1.003*             \\
                                                 &                    &                    & (0.023)            \\
Rating\_Entropy\_Mkt$^2$     &                    &                    & -1.079*            \\
                                                 &                    &                    & (0.019)            \\
Average\_Rating\_Mkt                             & -0.159             &                    &                    \\
                                                 & (0.252)            &                    &                    \\
Average\_Sentiment\_Mkt                          &                    & -1.310**           & -1.303**           \\
                                                 &                    & (0.008)            & (0.008)            \\
Average\_Experience                                       & 0.0139             & 0.0108             & 0.00768            \\
                                                 & (0.119)            & (0.213)            & (0.365)            \\
Price\_Difference\_Mkt                                & 0.388              & 0.411*             & 0.425*             \\
                                                 & (0.055)            & (0.035)            & (0.043)            \\
Market\_LogRevenue                               & -0.0357            & -0.0289            & -0.0271            \\
                                                 & (0.423)            & (0.504)            & (0.526)            \\
Text\_Dispersion\_Mkt                            & 20.58**            & 16.65*             & 14.67*             \\
                                                 & (0.005)            & (0.021)            & (0.047)            \\
Text\_Dispersion\_Mkt$^2$ & -73.10**           & -62.38**           & -57.62*            \\
                                                 & (0.002)            & (0.008)            & (0.016)            \\
Constant                                         & 2.174              & 2.553              & 2.517              \\
                                                 & (0.143)            & (0.053)            & (0.054)            \\
Adjusted R$^2$                                               & 0.745              & 0.749              & 0.752              \\
AIC                                              & 1077.6             & 1067.4             & 1064.6             \\
BIC                                              & 1166.8             & 1156.6             & 1162.7             \\ \bottomrule
\end{tabular}%
\begin{tablenotes}
\item Note: $p$-value in parentheses; $^\star p<0.05;^{\star\star} p<0.01;^{\star\star\star} p<0.001 $
\item The text-based dispersion variables are derived by taking the mean, instead of median, of the pairwise cosine distances.
\end{tablenotes}
\end{threeparttable}
\vspace{10pt}
\caption{Robustness Check - Market Level Analysis with Text-based Dispersion from Mean Cosine Distances}
\label{reg_mkt_rob_text_mean}
\end{table} 
%		% Please add the following required packages to your document preamble:
% \usepackage{booktabs}
% \usepackage{graphicx}
\begin{table}[]
\centering
\begin{threeparttable}[t]
\begin{tabular}{@{}lccc@{}}
\toprule
                                                     & (I)            & (II)           & (III)          \\ 
                                                     & Installer's    & Installer's    & Installer's    \\
Variables                                            & Activity Level & Activity Level & Activity Level \\ \midrule
Rating\_Entropy\_Self                                &                &                & 1.428***       \\
                                                     &                &                & (0.000)        \\
Rating\_Entropy\_Self$^2$                            &                &                & -2.590***      \\
                                                     &                &                & (0.000)        \\
Rating\_Entropy\_Others                              &                & 0.386          & 0.358          \\
                                                     &                & (0.085)        & (0.108)        \\
Rating\_Entropy\_Others$^2$                          &                & -2.243***      & -2.312***      \\
                                                     &                & (0.000)        & (0.000)        \\
Average\_Rating\_Self                                & -0.551**       & -0.524*        & -0.489*        \\
                                                     & (0.007)        & (0.011)        & (0.033)        \\
Average\_Rating\_Others                              & 0.00733        & 0.000713       & 0.0000629      \\
                                                     & (0.741)        & (0.977)        & (0.998)        \\
Review\_Count                                        & 0.0504***      & 0.0479***      & 0.0436***      \\
                                                     & (0.000)        & (0.000)        & (0.000)        \\
Experience                                           & 0.136*         & 0.134*         & 0.133*         \\
                                                     & (0.010)        & (0.012)        & (0.012)        \\
Price\_Difference                                    & 0.188*         & 0.200*         & 0.208*         \\
                                                     & (0.036)        & (0.027)        & (0.021)        \\
Market\_LogRevenue                                   & -0.00645       & -0.00618       & -0.00553       \\
                                                     & (0.148)        & (0.165)        & (0.216)        \\
Constant                                             & 2.788***       & 2.868***       & 2.995***       \\
                                                     & (0.000)        & (0.000)        & (0.000)        \\
Fixed Effect                                         & Yes            & Yes            & Yes            \\
State Dummies                                        & Yes            & Yes            & Yes            \\
Observations                                         & 3472           & 3472           & 3472           \\
Adjusted R$^2$                                          & 0.622          & 0.625          & 0.627          \\
AIC                                                  & 9693.2         & 9670.1         & 9655.4         \\
BIC                                                  & 9748.6         & 9737.7         & 9735.3         \\ \bottomrule
\end{tabular}%
\begin{tablenotes}
\item Note: $p$-value in parentheses; $^\star p<0.05;^{\star\star} p<0.01;^{\star\star\star} p<0.001 $
\end{tablenotes}
\caption{Robustness Check Excluding Inactive Installers}
\end{threeparttable}
\label{rob_exclude_inactive}
\end{table}
		
		
		%\section{Tests for Fixed and Random Effects in \eqref{model_ind_3}}\ref{Apx: Hausman}
		%
		% Individual Level - FE vs RE Hausman Test\\
		% $H_o$:  difference in coefficients not systematic
		%\begin{align*}
		%\chi^{2}(13) = (b-B)'[(V_b-V_B)^(-1)](b-B)=44.23\\
		%Prob>\chi^{2} =0.0000
		%\end{align*}
		%Market Level -FE vs. RE Hausman Test\\
		% $H_o$:  difference in coefficients not systematic
		%\begin{align*}
		%\chi^{2}(18) = (b-B)'[(V_b-V_B)^(-1)](b-B)=      403.30\\
		%Prob>\chi^{2} =0.0000
		%\end{align*}
		%
		%\label{Apx: Hausman}
		%\textbf{PLEASE INSERT A TABLE THAT REPORTS THE TEST RESULTS}
		
		
		%
		%\section{My Earlier Question}
		%
		%What is the total number of proposals in each year?
		%Which state is number \#1 in terms of total wins/total proposals?
		%Which state is worst in terms of total wins/total proposals?
		%Q1) IF THEY OPERATE AT MULTIPLE LOCATIONS, DO THEY PROVIDE
		%THAT INFO ON THEIR PROFILE? \\
		%We do not have info if they operated on multiple locations or not. I scrape their headquarter address , with ZIP code info.
		%Q2) WHAT IS THE FORMAT OF THE LOCATION INFO - IS IT A DETAILED
		%ONE WITH A ZIPCODE? PLEASE INCLUDE AN EXAMPLE FOR ME HERE. \\
		%Emerald Energy of North Carolina Headquarters
		%2624 Leighton Ridge Drive, Suite 120
		%Wake Forest, NC
		%27587 US
		%
		%COULD YOU PLEASE PREPARE THESE TWO GRAPHS? 1) TOTAL NUMBER
		%OF REVIEWS PER INSTALLER - MAX NUMBER OF REVIEW FOR AN IN-
		%STALLER AND HISTOGRAM? 2) NUMBER OF INSTALLERS IN EACH STATE
		%FOR TOP 10 STATES \\
		%QUESTIONS: 1) HOW MANY OF THE INSTALLERS DO NOT HAVE A RE-
		%VIEW? 2) WHAT IS THE PERCENTAGE OF THOSE INSTALLERS IN THE LO-
		%CAL MARKET WIN AND SUBMITTED PROPOSALS? 3) WHAT DO YOU AS-
		%SUME ABOUT THEM IN THE EMPIRICAL ANALYSIS? \\
		%We didn't need to assume anything. I just computed all the variables the way as we stated. \\
		%All installers started with 0 reviews, naturally. \\
		%If we look at the observations that are included in the analysis, less than 5 percent of observations have 0 reviews ( mostly due to its newly established). \\
		%If we look at the end of the panel, only 1 installer has 0 reviews, and have a positive entothers value ( hence is included in the analysis)
		%\textbf{YOU MENTIONED controls are to capture factors that are irrelevant to the rating entropy. Is Experience really irrelevant to the rating entropy?? }
		
%\end{APPENDICES}
	
	
	\clearpage
	% Appendix here
	% Options are (1) APPENDIX (with or without general title) or
	%             (2) APPENDICES (if it has more than one unrelated sections)
	% Outcomment the appropriate case if necessary
	%
	% \begin{APPENDIX}{<Title of the Appendix>}
	% \end{APPENDIX}
	%
	%   or
	%
	% \begin{APPENDICES}
	% \section{<Title of Section A>}
	% \section{<Title of Section B>}
	% etc
	% \end{APPENDICES}
	
	
	% Acknowledgments here
	\ACKNOWLEDGMENT{ .}

	
	
	
	
	
	%%%%%%%%%%%%%%%%%
\end{document}
%%%%%%%%%%%%%%%%%

