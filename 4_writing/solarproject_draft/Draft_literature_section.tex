\section{Literature and Theory Development }
 
\subsection{Literature on Reviews' variations    }
Prior to developing our hypotheses, we will briefly review the current literature on product reviews and their impact.  
The broader issue of product reviews has been addressed in current literature. Following the availability of data,  \cite{chevalier2006effect} found a positive relationship between reviews and product sales in publishing, \cite{wu2015economic} quantified the economic value per piece of positive reviews in helping consumers choosing restaurants, \cite{chintagunta2010effects} extended the study of reviews to movie box office, and \cite{xu2016interplay} furthered our understanding of reviews' role in physician choice. 

Within the literature on the \textbf{variations} of reviews, existing literature has largely studied from the perspective of product and their \textbf{own} reviews' variations. \cite{sun2012does} draw from the consumer utility theory with a Hoteling framework and a DID empirical setting on book sales; they found that higher variance of ratings could increase demand via its signaling effect. \cite{yi2019leveraging} studied the moderating effect of the variance of review... \cite{wang2015user} found that in multiple settings (movie box office, books  digital cameras and lab setting) high user reviews variance can be a double edged sword, and it could be positive if user reviews variance elicit a sense of uniqueness , or negative if it signaled poor quality. \cite{west1998integrating} linked box office response to reviews consensus to consumer aspirations which highlighted opposite effect of reviews variances.  

Another way to understand the impact of reviews on installers activities is from the perspective of performance reviews.  Product ratings is a form of performance reviews on the product or services that sellers/suppliers provided. In the stream of performance reviews literature, \cite{song2017closing} found public performance feedback to encourage low-performing physicians, while \cite{eyring2018performance} found a more nuanced impact of performance reviews in a classroom setting, citing that the reference point matters;  \cite{tan2019you} found a non-linear impact of coworker performance reviews. 
A corollary question to the studies of performance feedback is the uncertainty in feedbacks. \cite{marinovic2015credibility} formulated a game-theoretic model that featured a principal giving noise-mixed performance feedback. \cite{bolton2019inflated} studied inflation in performance feedbacks on the AirBnB setting. This question is related to our study of the variation on reviews. 