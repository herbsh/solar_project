
\subsection{Alternative Inverted-U Testing: Spline Regression}
Up to this point, we draw our conclusion about the inverted U-shape relationship based on a commonly used criterion - the significance of the positive sign associated with the linear term of $Entropy$ and the negative sign on the quadratic term. To provide further validation, we followed literature such as \cite{tan2019you,kesavan2014volume} with a spline regression to perform robustness checks with both individual and market level analysis. 
Spline regressions use knots to capture the changes in coefficients for different intervals of the independent variables. We performed both 2 and 3 knots for the spline regressions. For 2 knots, we created a collection of variables $\text{Rating\_Entropy\_Others\_1}$ ,$\text{Rating\_Entropy\_Others\_2}$   each containing a linear spline of the $\text{Rating\_Entropy\_Others}$ variable, splitting in the middle of the range of the variable. We plug in the collection of spline variables in place of the linear and quadratic term. Likewise, we create spline variables for $\text{Rating\_Entropy\_Self}$ and in the market level analysis, spline variables for $\text{Rating\_Entropy\_Mkt}$. We also used two breakpoints to create a collection of three spline variables. 

The results are presented in table \ref{rob_spline_ind} and \ref{rob_spline_mkt}. In table \ref{rob_spline_ind} we report the regression with  $\text{Rating\_Entropy\_Others}$ with two knots in column (1) and  $\text{Rating\_Entropy\_Self}$ in column (2). We find that the coefficient of the first spline is positive and significant while the second is negative and significant ($p<0.001)$ supporting the inverted-U relationships between either Entropy measure and activity levels. The conclusion are similar when we consider the case with 3 knots as shown in column (3) and (4).  Next we consider the market level analysis with results presented in \ref{rob_spline_mkt}. The positive and negative significant($p<0.001$) coefficients associated the two splines column (1) further validated the inverted-U relationship we established. When move the 3 knot case, the coefficient on the 3rd spline is insignificant which could be attributed to the sample size. 


% Please add the following required packages to your document preamble:
% \usepackage{booktabs}
% \usepackage{graphicx}
\begin{table}[]
\centering
\begin{threeparttable}[t]
\begin{tabular}{@{}lcccc@{}}
\toprule
                           & (I)            & (II)           & (III)          & (IV)           \\ 
                           & Installer's    & Installer's    & Installer's    & Installer's    \\
Variables                  & Activity Level & Activity Level & Activity Level & Activity Level \\ \midrule
Rating\_Entropy\_Others\_1 & 0.512*    &           &          &          \\
                           & (0.013)   &           &          &          \\
Rating\_Entropy\_Others\_2 & -1.934*** &           &          &          \\
                           & (0.000)   &           &          &          \\
Rating\_Entropy\_Self\_1   &           & 1.200***  &          &          \\
                           &           & (0.000)   &          &          \\
Rating\_Entropy\_Self\_2   &           & -3.072*** &          &          \\
                           &           & (0.000)   &          &          \\
Rating\_Entropy\_Others\_1 &           &           & 0.834*** &          \\
                           &           &           & (0.001)  &          \\
Rating\_Entropy\_Others\_2 &           &           & -1.164*  &          \\
                           &           &           & (0.017)  &          \\
Rating\_Entropy\_Others\_3 &           &           & -1.763*  &          \\
                           &           &           & (0.020)  &          \\
Rating\_Entropy\_Self\_1   &           &           &          & 1.746*** \\
                           &           &           &          & (0.000)  \\
Rating\_Entropy\_Self\_2   &           &           &          & -1.015   \\
                           &           &           &          & (0.099)  \\
Rating\_Entropy\_Self\_3   &           &           &          & -4.641** \\
                           &           &           &          & (0.008)  \\
Observations               & 4562      & 4562      & 4562     & 4562     \\
Adjusted R$^2$                         & 0.631     & 0.633     & 0.632    & 0.633    \\
AIC                       & 13222.5   & 13204.4   & 13218.4  & 13203.5  \\
BIC                        & 13306.0   & 13287.9   & 13308.3  & 13293.4 \\ \bottomrule
\end{tabular}%
\begin{tablenotes}
\item Note: $p$-value in parentheses; $^\star p<0.05;^{\star\star} p<0.01;^{\star\star\star} p<0.001 $
\end{tablenotes}
\end{threeparttable}
\caption{Alternative Inverted-U Testing: Spline Regressions (Installer Level)}
\label{rob_spline_ind}
\end{table}
% Please add the following required packages to your document preamble:
% \usepackage{booktabs}
% \usepackage{graphicx}
\begin{table}[]
\centering
 \begin{threeparttable}[t]
\begin{tabular}{@{}lcc@{}}
\toprule
                        & (I)            & (II)           \\ 
                        & Market Transaction       & Market Transaction       \\
Variables               &   &   \\ \midrule
Rating\_Entropy\_Mkt\_1 & 1.197***       &                \\
                        & (0.000)        &                \\
Rating\_Entropy\_Mkt\_2 & -1.144**       &                \\
                        & (0.008)        &                \\
Rating\_Entropy\_Mkt\_1 &                & 2.157***       \\
                        &                & (0.000)        \\
Rating\_Entropy\_Mkt\_2 &                & -2.100***      \\
                        &                & (0.000)        \\
Rating\_Entropy\_Mkt\_3 &                & 0.356          \\
                        &                & (0.606)        \\
Observations            & 642            & 642            \\
Adjusted R$^2$             & 0.720          & 0.732          \\
AIC                     & 1101.0         & 1074.3         \\
BIC                     & 1136.7         & 1114.4         \\ \bottomrule
\end{tabular}%
\begin{tablenotes}
\item Note: $p$-value in parentheses; $^\star p<0.05;^{\star\star} p<0.01;^{\star\star\star} p<0.001 $
\end{tablenotes}
\end{threeparttable}
\caption{Alternative Inverted-U Testing: Spline Regressions (Market Level)}
\label{rob_spline_mkt}
\end{table}